\section{Выполнение}

\subsection{Этап 1. Резервное копирование}

Для выполнения работы был использован скрипт, который выполняет следующие действия:

\begin{itemize}
    \item Создание временной метки для имени резервной копии в формате год-месяц-день\_час:минута:секунда
    \item Настройка директории для хранения резервных копий (/var/db/postgres1/backup)
    \item Создание резервной копии PostgreSQL с помощью утилиты pg\_basebackup
    \item Архивация резервной копии в tar-файл с применением сжатия
    \item Передача архива на удалённый сервер (postgres2@pg184) через scp
    \item Удаление устаревших резервных копий на локальной машине
    \item Удаление устаревших резервных копий на удалённой машине
    \item Обработка ошибок на каждом этапе для обеспечения целостности процесса резервного копирования
\end{itemize}

\inputminted[breaklines,linenos,frame=single]{bash}{backup.sh}

Пример выполнения скрипта:

\begin{verbatim}
[postgres1@pg183 ~]$ ./backup.sh
Starting backup...
Creating backup using pg_basebackup...
Archiving backup directory into tar file...
Sending backup files to remote server...
2025-04-27_20:26:14.tar.gz                                                                                                                             100% 3935KB 139.2MB/s   00:00
Deleting old backups on host machine
Deleting old backups on remote machine
Backup created successfully: /var/db/postgres1/backup/2025-04-27_20:26:14!
[postgres1@pg183 ~]$ ls -lah backup/
total 19782
drwxr-xr-x  7 postgres1 postgres   12B 27 апр.  20:26 .
drwxr-xr-x  7 postgres1 postgres   12B 27 апр.  20:10 ..
drwxr-xr-x  2 postgres1 postgres    6B 27 апр.  20:19 2025-04-27_20:19:10
-rw-r--r--  1 postgres1 postgres  3,8M 27 апр.  20:19 2025-04-27_20:19:10.tar.gz
drwxr-xr-x  2 postgres1 postgres    6B 27 апр.  20:20 2025-04-27_20:20:46
-rw-r--r--  1 postgres1 postgres  3,8M 27 апр.  20:20 2025-04-27_20:20:46.tar.gz
drwxr-xr-x  2 postgres1 postgres    6B 27 апр.  20:25 2025-04-27_20:25:12
-rw-r--r--  1 postgres1 postgres  3,8M 27 апр.  20:25 2025-04-27_20:25:12.tar.gz
drwxr-xr-x  2 postgres1 postgres    6B 27 апр.  20:26 2025-04-27_20:26:06
-rw-r--r--  1 postgres1 postgres  3,8M 27 апр.  20:26 2025-04-27_20:26:06.tar.gz
drwxr-xr-x  2 postgres1 postgres    6B 27 апр.  20:26 2025-04-27_20:26:14
-rw-r--r--  1 postgres1 postgres  3,8M 27 апр.  20:26 2025-04-27_20:26:14.tar.gz
\end{verbatim}

Скопированные данные на `pg184':

\begin{verbatim}
[postgres2@pg184 ~]$ ls -lah backup/
total 19732
drwxr-xr-x  2 postgres2 postgres    7B 27 апр.  20:26 .
drwxr-xr-x  4 postgres2 postgres    5B 27 апр.  20:18 ..
-rw-r--r--  1 postgres2 postgres  3,8M 27 апр.  20:19 2025-04-27_20:19:10.tar.gz
-rw-r--r--  1 postgres2 postgres  3,8M 27 апр.  20:20 2025-04-27_20:20:46.tar.gz
-rw-r--r--  1 postgres2 postgres  3,8M 27 апр.  20:25 2025-04-27_20:25:12.tar.gz
-rw-r--r--  1 postgres2 postgres  3,8M 27 апр.  20:26 2025-04-27_20:26:06.tar.gz
-rw-r--r--  1 postgres2 postgres  3,8M 27 апр.  20:26 2025-04-27_20:26:14.tar.gz
\end{verbatim}

Для запуска резервирования по расписанию был использован cron.

\begin{verbatim}
0 0 * * * $HOME/backup.sh >> $HOME/backup.log 2>&1
0 12 * * * $HOME/backup.sh >> $HOME/backup.log 2>&1
\end{verbatim}

\subsubsection{Расчет объема резервных копий}

Для расчета объема резервных копий использованы следующие исходные данные:
\begin{itemize}
    \item Средний объем новых данных в БД за сутки: 50 МБ
    \item Средний объем измененных данных за сутки: 950 МБ
    \item Частота резервного копирования: 2 раза в сутки
    \item Срок хранения на резервной системе: 1 месяц (30 дней)
\end{itemize}

Расчет объема:
\begin{itemize}
    \item Объем одной копии: 50 МБ + 950 МБ = 1000 МБ = 1 ГБ
    \item Количество копий за месяц: 2 копии/день $\times$ 30 дней = 60 копий
    \item Общий объем за месяц: 60 копий $\times$ 1 ГБ = 60 ГБ
\end{itemize}

Таким образом, через месяц работы системы на резервном узле будет храниться 60 ГБ резервных копий. Это расчетное значение, которое может варьироваться в зависимости от реального объема изменений в базе данных.

\subsection{Этап 2. Потеря основного узла}

Этот сценарий подразумевает полную недоступность основного узла. Необходимо восстановить работу СУБД на РЕЗЕРВНОМ узле, продемонстрировать успешный запуск СУБД и доступность данных.

\inputminted[breaklines,linenos,frame=single]{bash}{restore.sh}

До выполнения скрипта:

\begin{verbatim}
[postgres1@pg183 ~]$ psql -h localhost -p 9115 postgres
psql (16.4)
Введите "help", чтобы получить справку.

postgres=# create table test(id serial);
CREATE TABLE
postgres=# insert into test(id) values (1), (2), (3);
INSERT 0 3
postgres=#
\q
\end{verbatim}

После выполнения скрипта на резервном узле:

\begin{verbatim}
[postgres2@pg184 ~]$ psql -h localhost -p 9115 -U postgres1 postgres
psql (16.4)
Введите "help", чтобы получить справку.

postgres=# \dt
            Список отношений
    Схема  | Имя  |   Тип   | Владелец
--------+------+---------+-----------
    public | test | таблица | postgres1
(1 строка)

postgres=# select * from test;
    id
----
    1
    2
    3
(3 строки)
\end{verbatim}

\subsection{Этап 3. Повреждение файлов БД}

Повреждение данных:
\begin{verbatim}
[postgres1@pg183 ~]$ rm -rf fkc19/pg_tblspc/16384/
\end{verbatim}

После удаления данных БД сохраняет работоспособность:

\begin{verbatim}
[postgres1@pg183 ~]$ psql -h localhost -p 9115 postgres
psql (16.4)
Введите "help", чтобы получить справку.

postgres=# \dt
            Список отношений
    Схема  | Имя  |   Тип   | Владелец
--------+------+---------+-----------
    public | test | таблица | postgres1
(1 строка)

postgres=# select * from test;
    id
----
    1
    2
    3
(3 строки)

postgres=#
\end{verbatim}

При этом в логах появляются сообщения о том, что каталог не найден.

\begin{verbatim}
[postgres1@pg183 ~]$ pg_ctl restart
ожидание завершения работы сервера.... готово
сервер остановлен
ожидание запуска сервера....2025-04-27 21:37:17.334 MSK [17052] СООБЩЕНИЕ:  передача вывода в протокол процессу сбора протоколов
2025-04-27 21:37:17.334 MSK [17052] ПОДСКАЗКА:  В дальнейшем протоколы будут выводиться в каталог "log".
    готово
сервер запущен
[postgres1@pg183 ~]$ pg_ctl status
pg_ctl: сервер работает (PID: 17052)
/usr/local/bin/postgres

[postgres1@pg183 ~]$ cat fkc19/log/postgresql-2025-04-27_213717.csv
2025-04-27 21:37:17.334 MSK,,,17052,,680e795d.429c,1,,2025-04-27 21:37:17 MSK,,0,СООБЩЕНИЕ,00000,"завершение вывода в stderr",,"В дальнейшем протокол будет выводиться в ""csvlog"".",,,,,,,"","postmaster",,0
2025-04-27 21:37:17.334 MSK,,,17052,,680e795d.429c,2,,2025-04-27 21:37:17 MSK,,0,СООБЩЕНИЕ,00000,"запускается PostgreSQL 16.4 on amd64-portbld-freebsd14.1, compiled by FreeBSD clang version 18.1.6 (https://github.com/llvm/llvm-project.git llvmorg-18.1.6-0-g1118c2e05e67), 64-bit",,,,,,,,,"","postmaster",,0
2025-04-27 21:37:17.334 MSK,,,17052,,680e795d.429c,3,,2025-04-27 21:37:17 MSK,,0,СООБЩЕНИЕ,00000,"для приёма подключений по адресу IPv6 ""::1"" открыт порт 9115",,,,,,,,,"","postmaster",,0
2025-04-27 21:37:17.334 MSK,,,17052,,680e795d.429c,4,,2025-04-27 21:37:17 MSK,,0,СООБЩЕНИЕ,00000,"для приёма подключений по адресу IPv4 ""127.0.0.1"" открыт порт 9115",,,,,,,,,"","postmaster",,0
2025-04-27 21:37:17.358 MSK,,,17052,,680e795d.429c,5,,2025-04-27 21:37:17 MSK,,0,СООБЩЕНИЕ,00000,"для приёма подключений открыт Unix-сокет ""/tmp/.s.PGSQL.9115""",,,,,,,,,"","postmaster",,0
2025-04-27 21:37:17.378 MSK,,,17052,,680e795d.429c,6,,2025-04-27 21:37:17 MSK,,0,СООБЩЕНИЕ,58P01,"не удалось открыть каталог ""pg_tblspc/16384/PG_16_202307071"": No such file or directory",,,,,,,,,"","postmaster",,0
2025-04-27 21:37:17.409 MSK,,,17057,,680e795d.42a1,1,,2025-04-27 21:37:17 MSK,,0,СООБЩЕНИЕ,00000,"система БД была выключена: 2025-04-27 21:37:17 MSK",,,,,,,,,"","startup",,0
2025-04-27 21:37:17.410 MSK,,,17057,,680e795d.42a1,2,,2025-04-27 21:37:17 MSK,,0,СООБЩЕНИЕ,58P01,"не удалось открыть каталог ""pg_tblspc/16384/PG_16_202307071"": No such file or directory",,,,,,,,,"","startup",,0
2025-04-27 21:37:17.440 MSK,,,17052,,680e795d.429c,7,,2025-04-27 21:37:17 MSK,,0,СООБЩЕНИЕ,00000,"система БД готова принимать подключения",,,,,,,,,"","postmaster",,0
\end{verbatim}

После восстановления данных с помощью скрипта в новой базе ошибка не воспроизводится.

\begin{verbatim}
[postgres1@pg183 ~]$ cat pgdata/log/postgresql-2025-04-27_214120.csv
2025-04-27 21:41:20.967 MSK,,,19297,,680e7a50.4b61,1,,2025-04-27 21:41:20 MSK,,0,СООБЩЕНИЕ,00000,"завершение вывода в stderr",,"В дальнейшем протокол будет выводиться в ""csvlog"".",,,,,,,"","postmaster",,0
2025-04-27 21:41:20.967 MSK,,,19297,,680e7a50.4b61,2,,2025-04-27 21:41:20 MSK,,0,СООБЩЕНИЕ,00000,"запускается PostgreSQL 16.4 on amd64-portbld-freebsd14.1, compiled by FreeBSD clang version 18.1.6 (https://github.com/llvm/llvm-project.git llvmorg-18.1.6-0-g1118c2e05e67), 64-bit",,,,,,,,,"","postmaster",,0
2025-04-27 21:41:20.968 MSK,,,19297,,680e7a50.4b61,3,,2025-04-27 21:41:20 MSK,,0,СООБЩЕНИЕ,00000,"для приёма подключений по адресу IPv6 ""::1"" открыт порт 9115",,,,,,,,,"","postmaster",,0
2025-04-27 21:41:20.968 MSK,,,19297,,680e7a50.4b61,4,,2025-04-27 21:41:20 MSK,,0,СООБЩЕНИЕ,00000,"для приёма подключений по адресу IPv4 ""127.0.0.1"" открыт порт 9115",,,,,,,,,"","postmaster",,0
2025-04-27 21:41:20.989 MSK,,,19297,,680e7a50.4b61,5,,2025-04-27 21:41:20 MSK,,0,СООБЩЕНИЕ,00000,"для приёма подключений открыт Unix-сокет ""/tmp/.s.PGSQL.9115""",,,,,,,,,"","postmaster",,0
2025-04-27 21:41:21.055 MSK,,,19302,,680e7a51.4b66,1,,2025-04-27 21:41:21 MSK,,0,СООБЩЕНИЕ,00000,"система БД была выключена: 2025-04-27 21:41:18 MSK",,,,,,,,,"","startup",,0
2025-04-27 21:41:21.073 MSK,,,19297,,680e7a50.4b61,6,,2025-04-27 21:41:20 MSK,,0,СООБЩЕНИЕ,00000,"система БД готова принимать подключения",,,,,,,,,"","postmaster",,0
\end{verbatim}

\subsection{Этап 4. Логирование повреждение данных}

Данные до повреждения:
\begin{verbatim}
[postgres1@pg183 ~]$ psql -h localhost -p 9115 postgres
psql (16.4)
Введите "help", чтобы получить справку.

postgres=# \dt
            Список отношений
    Схема  |  Имя  |   Тип   | Владелец
--------+-------+---------+-----------
    public | test  | таблица | postgres1
    public | test1 | таблица | postgres1
(2 строки)

postgres=# select * from test;
    id
----
    1
    2
    3
(3 строки)

postgres=# select * from test1;
    id | key
----+-----
    1 |   1
    2 |   2
    3 |   3
(3 строки)
\end{verbatim}

После повреждения данных:
\begin{verbatim}
[postgres1@pg183 ~]$ psql -h localhost -p 9115 postgres
psql (16.4)
Введите "help", чтобы получить справку.

postgres=# update test1 set key = 1;
UPDATE 3
postgres=# select * from test1;
id | key
----+-----
1 |   1
2 |   1
3 |   1
(3 строки)
\end{verbatim}

Для восстановления данных создадим дамп на резервном узле:

\begin{verbatim}
[postgres2@pg184 ~]$ pg_dump -Fc -U postgres1 -h localhost -p 9115 postgres -f pgdump
\end{verbatim}

Перенесем дамп но основной узел:

\begin{verbatim}
[postgres1@pg183 ~]$ scp postgres2@pg184:~/pgdump
pgdump.sql       100% 3025     2.5MB/s   00:00
\end{verbatim}

Восстановим данные в новой базе:

\begin{verbatim}
[postgres1@pg183 ~]$ dropdb -h localhost -p 9115 postgres
[postgres1@pg183 ~]$ createdb -h localhost -p 9115 postgres
[postgres1@pg183 ~]$ pg_restore -h localhost -p 9115 -d postgres pgdump
\end{verbatim}

После восстановления данных:

\begin{verbatim}
[postgres1@pg183 ~]$ psql -h localhost -p 9115 postgres
psql (16.4)
Введите "help", чтобы получить справку.

postgres=# \d
                    Список отношений
    Схема  |     Имя      |        Тип         | Владелец
--------+--------------+--------------------+-----------
    public | test         | таблица            | postgres1
    public | test1        | таблица            | postgres1
    public | test1_id_seq | последовательность | postgres1
    public | test_id_seq  | последовательность | postgres1
(4 строки)

postgres=# \dt
            Список отношений
    Схема  |  Имя  |   Тип   | Владелец
--------+-------+---------+-----------
    public | test  | таблица | postgres1
    public | test1 | таблица | postgres1
(2 строки)

postgres=# select * from test;
    id
----
    1
    2
    3
(3 строки)

postgres=# select * from test1;
    id | key
----+-----
    1 |   1
    2 |   2
    3 |   3
(3 строки)
\end{verbatim}