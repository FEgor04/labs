\section{Вывод}

В ходе выполнения лабораторной работы была успешно реализована система резервного копирования и восстановления базы данных PostgreSQL. Основные достижения включают:

\begin{itemize}
    \item Настроено автоматическое резервное копирование с помощью скрипта \texttt{backup.sh}, который выполняет полное копирование базы данных два раза в сутки через cron.
    \item Реализована система хранения резервных копий с разными сроками хранения: 1 неделя на основном узле и 1 месяц на резервном.
    \item Проведен расчет объема резервных копий, который показал, что при заданных условиях (50 МБ новых данных и 950 МБ измененных данных в сутки) через месяц работы системы потребуется 60 ГБ дискового пространства.
    \item Успешно протестированы три сценария восстановления:
    \begin{itemize}
        \item Восстановление при полной потере основного узла
        \item Восстановление при повреждении файлов базы данных
        \item Восстановление при логическом повреждении данных
    \end{itemize}
\end{itemize}

Практическая часть работы продемонстрировала эффективность реализованных механизмов резервного копирования и восстановления. Все тестовые сценарии были успешно выполнены, что подтверждает надежность и работоспособность системы.

Особенно важно отметить, что система способна восстанавливать данные даже в сложных ситуациях, таких как:
\begin{itemize}
    \item Полная потеря основного узла
    \item Повреждение файлов табличных пространств
    \item Логическое повреждение данных в результате ошибочных операций
\end{itemize}

Результаты работы показали, что реализованная система резервного копирования и восстановления обеспечивает надежную защиту данных и позволяет быстро восстанавливать работоспособность базы данных в случае различных сбоев.
