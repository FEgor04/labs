\section{Задание}
Вариант 368121.
Цель работы - настроить процедуру периодического резервного копирования базы данных, сконфигурированной в ходе выполнения лабораторной работы №2, а также разработать и отладить сценарии восстановления в случае сбоев.

Узел из предыдущей лабораторной работы используется в качестве основного. Новый узел используется в качестве резервного. Учётные данные для подключения к новому узлу выдаёт преподаватель. В сценариях восстановления необходимо использовать копию данных, полученную на первом этапе данной лабораторной работы.

\subsection{Требования к отчёту}
Отчет должен быть самостоятельным документом (без ссылок на внешние ресурсы), содержать всю последовательность команд и исходный код скриптов по каждому пункту задания. Для демонстрации результатов приводить команду вместе с выводом (самой наглядной частью вывода, при необходимости).

\subsection{Резервное копирование}
Настроить резервное копирование с основного узла на резервный следующим образом:
Периодические обособленные (standalone) полные копии.
Полное резервное копирование (pg\_basebackup) по расписанию (cron) два раза в сутки. Необходимые файлы WAL должны быть в составе полной копии, отдельно их не архивировать. Срок хранения копий на основной системе - 1 неделя, на резервной - 1 месяц. По истечении срока хранения, старые архивы должны автоматически уничтожаться.

Подсчитать, каков будет объем резервных копий спустя месяц работы системы, исходя из следующих условий:

\begin{itemize}
    \item Средний объем новых данных в БД за сутки: 50МБ.
    \item Средний объем измененных данных за сутки: 950МБ.
\end{itemize}

\subsection{Потеря основного узла}
Этот сценарий подразумевает полную недоступность основного узла.
Необходимо восстановить работу СУБД на РЕЗЕРВНОМ узле,
продемонстрировать успешный запуск СУБД и доступность данных.

\subsection{Повреждение файлов БД}
Этот сценарий подразумевает потерю данных (например, в результате сбоя диска или файловой системы) при сохранении доступности основного узла. Необходимо выполнить полное восстановление данных из резервной копии и перезапустить СУБД на ОСНОВНОМ узле.

Ход работы

\begin{itemize}
\item Симулировать сбой: удалить с диска директорию любого табличного пространства со всем содержимым.
\item Проверить работу СУБД, доступность данных, перезапустить СУБД, проанализировать результаты.
\item Выполнить восстановление данных из резервной копии, учитывая следующее условие: исходное расположение дополнительных табличных пространств недоступно - разместить в другой директории и скорректировать конфигурацию.
\item Запустить СУБД, проверить работу и доступность данных, проанализировать результаты.
\end{itemize}


\subsection{Логическое повреждение данных}
Этот сценарий подразумевает частичную потерю данных (в результате нежелательной или ошибочной операции)
при сохранении доступности основного узла. Необходимо выполнить восстановление данных на ОСНОВНОМ узле следующим способом: Генерация файла на резервном узле с помощью pg\_dump и последующее применение файла на основном узле.

Ход работы:

\begin{itemize}
    \item В каждую таблицу базы добавить 2-3 новые строки, зафиксировать результат.
    \item Зафиксировать время и симулировать ошибку: в любой таблице с внешними ключами подменить значения ключей на случайные (INSERT, UPDATE)
    \item Продемонстрировать результат.
    \item Выполнить восстановление данных указанным способом.
    \item Продемонстрировать и проанализировать результат.
\end{itemize}

