\section{Выполнение}

\begin{enumerate}
\item \textbf{Инициализация кластера БД}

Подключимся к выделенному узлу через SSH:

\begin{verbatim}
ssh -J s367581@helios.cs.ifmo.ru:2222 postgres1@pg183
\end{verbatim}

Создадим директорию для кластера:

\begin{verbatim}
mkdir -p $HOME/fkc19
\end{verbatim}

Зададим переменные окружения для инициализации кластера с русской локалью и кодировкой ISO\_8859\_5:

\begin{verbatim}
export PGDATA=$HOME/fkc19
export LOCALE=ru_RU.ISO8859-5
export PGENCODING=ISO_8859_5
export PG_USERNAME=postgres1
\end{verbatim}

\begin{verbatim}
[postgres1@pg183 ~]$ env | sort
BLOCKSIZE=K
HOME=/var/db/postgres1
LANG=ru_RU
LOCALE=ru_RU.ISO8859-5
LOGNAME=postgres1
MAIL=/var/mail/postgres1
MM_CHARSET=UTF-8
PATH=/sbin:/bin:/usr/sbin:/usr/bin:/usr/local/sbin:/usr/local/bin:/var/db/postgres1/bin
PGDATA=/var/db/postgres1/fkc19
PGENCODING=ISO_8859_5
PG_USERNAME=postgres1
PWD=/var/db/postgres1
SHELL=/usr/local/bin/bash
SHLVL=1
SSH_CLIENT=192.168.10.80 37422 22
SSH_CONNECTION=192.168.10.80 37422 192.168.11.183 22
SSH_TTY=/dev/pts/58
TERM=xterm-256color
USER=postgres1
_=/usr/bin/env
\end{verbatim}


Инициализируем кластер БД с использованием переменных окружения:

\begin{verbatim}
[postgres1@pg183 ~]$ initdb --encoding=$PGENCODING        --locale=$LOCALE        --username=$PG_USERNAME
The files belonging to this database system will be owned by user "postgres1".
This user must also own the server process.

The database cluster will be initialized with locale "ru_RU.ISO8859-5".
The default text search configuration will be set to "russian".

Data page checksums are disabled.

creating directory /var/db/postgres1/fkc19 ... ok
creating subdirectories ... ok
selecting dynamic shared memory implementation ... posix
selecting default max_connections ... 100
selecting default shared_buffers ... 128MB
selecting default time zone ... Europe/Moscow
creating configuration files ... ok
running bootstrap script ... ok
performing post-bootstrap initialization ... ok
syncing data to disk ... ok

initdb: warning: enabling "trust" authentication for local connections
initdb: hint: You can change this by editing pg_hba.conf or using the option -A, or --auth-local and --auth-host, the next time you run initdb.

Success. You can now start the database server using:

    pg_ctl -D /var/db/postgres1/fkc19 -l logfile start

\end{verbatim}

\item \textbf{Конфигурация сервера БД}

Для настройки сервера отредактируем конфигурационный файл \texttt{postgresql.conf}:

\begin{verbatim}
vim $PGDATA/postgresql.conf
\end{verbatim}

Внесем следующие изменения:

\begin{verbatim}
# Настройка подключений
listen_addresses = 'localhost'      # только локальные подключения
port = 9115                         # номер порта
unix_socket_directories = '/tmp'    # директория для Unix-сокета

# Настройка производительности для OLTP (1500 транзакций/сек по 16КБ)
max_connections = 200               # максимальное число соединений
shared_buffers = 1GB                # размер общего буфера (25% от RAM)
temp_buffers = 16MB                 # буферы для временных таблиц
work_mem = 32MB                     # память для операций сортировки
effective_cache_size = 3GB          # оценка доступной памяти для кэша
fsync = on                          # обеспечение надежности данных (HA)
checkpoint_timeout = 5min           # время между контрольными точками
commit_delay = 1000                 # задержка фиксации (в микросекундах)

# Настройка WAL
wal_level = replica                 # уровень WAL для обеспечения HA
wal_directory = 'pg_wal'            # директория для WAL файлов

# Настройка логирования
log_destination = 'csvlog'          # формат лог-файлов
logging_collector = on              # включение сборщика логов
log_directory = 'log'               # директория для логов
log_filename = 'postgresql-%Y-%m-%d_%H%M%S.log'
log_min_messages = WARNING          # минимальный уровень сообщений
log_connections = on                # логирование подключений
log_disconnections = on             # логирование отключений
log_duration = on                   # логирование продолжительности выполнения
\end{verbatim}

Отредактируем файл pg\_hba.conf для настройки аутентификации:

\begin{verbatim}
vim $PGDATA/pg_hba.conf
\end{verbatim}

Заменим содержимое на следующее:

\begin{verbatim}
# TYPE  DATABASE        USER            ADDRESS                 METHOD

# Unix-domain socket connections
local   all             all                                     peer
# TCP/IP connections from localhost only with SHA-256 password
host    all             all             127.0.0.1/32            scram-sha-256
host    all             all             ::1/128                 scram-sha-256
\end{verbatim}

\item \textbf{Запуск сервера БД}

Запустим сервер PostgreSQL:

\begin{verbatim}
pg_ctl -D $PGDATA start
\end{verbatim}

Проверим статус сервера:

\begin{verbatim}
pg_ctl -D $PGDATA status
\end{verbatim}

\item \textbf{Создание табличного пространства для индексов}

Создадим директорию для нового табличного пространства:

\begin{verbatim}
mkdir -p $HOME/uhx43
chmod 700 $HOME/uhx43
\end{verbatim}

Подключимся к серверу PostgreSQL:

\begin{verbatim}
psql -h localhost -p 9115 postgres
\end{verbatim}

Создадим табличное пространство для индексов:

\begin{verbatim}
CREATE TABLESPACE index_space LOCATION '/var/db/postgres1/uhx43';
\end{verbatim}

\begin{verbatim}
[postgres1@pg183 ~/uhx43]$ psql -h localhost -p 9115 postgres
psql (16.4)
Введите "help", чтобы получить справку.

postgres=# CREATE TABLESPACE index_space LOCATION '/var/db/postgres1/uhx43';
CREATE TABLESPACE
\end{verbatim}

\item \textbf{Создание новой базы данных}

Создадим новую базу данных на основе template1:

\begin{verbatim}
postgres=# CREATE DATABASE fatgreeenidea TEMPLATE template1;
CREATE DATABASE
\end{verbatim}

\item \textbf{Создание новой роли}

Создадим новую роль с правом на подключение к базе:

\begin{verbatim}
postgres=# CREATE ROLE app_user WITH LOGIN PASSWORD 'secure_password';
CREATE ROLE
postgres=# GRANT CONNECT ON DATABASE fatgreeenidea TO app_user;
GRANT
\end{verbatim}

Подключимся к базе данных fatgreenidea и создадим необходимые схемы и разрешения:

\begin{verbatim}
\c fatgreenidea
CREATE SCHEMA app_schema;
GRANT USAGE ON SCHEMA app_schema TO app_user;
GRANT CREATE ON SCHEMA app_schema TO app_user;
\end{verbatim}

\item \textbf{Наполнение базы тестовыми данными}

Подключимся к базе данных от имени созданной роли:

\begin{verbatim}
psql -h localhost -p 9115 -U app_user fatgreenidea
\end{verbatim}

Создадим таблицы, указав для индексов использование созданного табличного пространства:

\begin{verbatim}
CREATE TABLE table1 (id SERIAL PRIMARY KEY, name TEXT);
INSERT INTO table1 (name) VALUES ('Test Name 1'), ('Test Name 2');
\end{verbatim}

\item \textbf{Просмотр табличных пространств и их объектов}

Выведем список всех табличных пространств:

\begin{verbatim}
SELECT * FROM pg_tablespace;

oid  |   spcname   | spcowner | spcacl | spcoptions
-------+-------------+----------+--------+------------
  1663 | pg_default  |       10 |        |
  1664 | pg_global   |       10 |        |
 16384 | index_space |       10 |        |
(3 строки)
\end{verbatim}

Выведем список объектов, хранящихся в каждом табличном пространстве:

\begin{verbatim}
SELECT
    relname, pg_tablespace.spcname
FROM pg_class
JOIN pg_tablespace ON pg_class.reltablespace = pg_tablespace.oid
WHERE pg_class.relkind = 'r';

relname        |  spcname
-----------------------+-----------
 pg_authid             | pg_global
 pg_subscription       | pg_global
 pg_database           | pg_global
 pg_db_role_setting    | pg_global
 pg_tablespace         | pg_global
 pg_auth_members       | pg_global
 pg_shdepend           | pg_global
 pg_shdescription      | pg_global
 pg_replication_origin | pg_global
 pg_shseclabel         | pg_global
 pg_parameter_acl      | pg_global
(11 строк)
\end{verbatim}

\end{enumerate}