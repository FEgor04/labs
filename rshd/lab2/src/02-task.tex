\section{Задание}
Цель работы - на выделенном узле создать и сконфигурировать новый кластер БД Postgres,
саму БД, табличные пространства и новую роль,
а также произвести наполнение базы в соответствии с заданием.
Отчёт по работе должен содержать все команды по настройке, скрипты,
а также измененные строки конфигурационных файлов.

Способ подключения к узлу из сети Интернет через helios:
\texttt{ssh -J\- sXXXXXX@helios.cs.ifmo\-.ru:2222 postgresY@pgZZZ}.

Способ подключения к узлу из сети факультета:
\texttt{ssh postgresY@pgZZZ}
Номер выделенного узла pgZZZ, а также логин и пароль для подключения Вам выдаст преподаватель.

\subsection{Инициализация кластера БД}
	\begin{itemize}
		\item Директория кластера: 
		\$HOME/fkc19
		\item 
		Кодировка: ISO\_8859\_5
		\item 
		Локаль: русская
		\item 
		Параметры инициализации задать через переменные окружения
	\end{itemize}

\subsection{Конфигурация и запуск сервера БД}
 \begin{itemize}
	\item Способы подключения: 
	\begin{itemize}
		\item 1) Unix-domain сокет в режиме peer; 
		\item 2) сокет TCP/IP, только localhost
	\end{itemize}
	\item Номер порта: 9115
	\item Способ аутентификации TCP/IP клиентов: по паролю SHA-256
	\item Остальные способы подключений запретить.
	\item Настроить следующие параметры сервера БД:
	\begin{itemize}
			\item max\_connections
			\item shared\_buffers
			\item temp\_buffers
			\item work\_mem
			\item checkpoint\_timeout
			\item effective\_cache\_size
			\item fsync
			\item commit\_delay
	\end{itemize}
	Параметры должны быть подобраны в соответствии со сценарием OLTP:
	1500 транзакций в секунду размером 16КБ; обеспечить высокую доступность (High Availability) данных.
	\item Директория WAL файлов: \$PGDATA/pg\_wal
	\item Формат лог-файлов: .csv
	\item Уровень сообщений лога: WARNING
	\item Дополнительно логировать: завершение сессий и продолжительность выполнения команд
\end{itemize}

\subsection{Дополнительные табличные пространства и наполнение базы}
\begin{itemize}
	\item Создать новое табличное пространство для индексов: 
	\$HOME/uhx43
	\item На основе template1 создать новую базу: fatgreenidea
	\item Создать новую роль, предоставить необходимые права, разрешить подключение к базе.
	\item От имени новой роли (не администратора) произвести наполнение ВСЕХ созданных баз тестовыми наборами данных. ВСЕ табличные пространства должны использоваться по назначению.
	\item Вывести список всех табличных пространств кластера и содержащиеся в них объекты.
\end{itemize}
