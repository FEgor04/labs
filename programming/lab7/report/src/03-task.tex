\section{Текст задания}

Доработать программу из лабораторной работы №6 следующим образом:
\begin{itemize}
  \item Организовать хранение коллекции в реляционной СУБД (PostgresQL). Убрать хранение коллекции в файле.
  \item Для генерации поля id использовать средства базы данных (sequence).
  \item Обновлять состояние коллекции в памяти только при успешном добавлении объекта в БД
  \item Все команды получения данных должны работать с коллекцией в памяти, а не в БД
  \item Организовать возможность регистрации и авторизации пользователей. У пользователя есть возможность указать пароль.
  \item Пароли при хранении хэшировать алгоритмом SHA-256
  \item Запретить выполнение команд не авторизованным пользователям.
  \item При хранении объектов сохранять информацию о пользователе, который создал этот объект.
  \item Пользователи должны иметь возможность просмотра всех объектов коллекции, но модифицировать могут только принадлежащие им.
  \item Для идентификации пользователя отправлять логин и пароль с каждым запросом.
\end{itemize}

Необходимо реализовать многопоточную обработку запросов.

\begin{itemize}
  \item Для многопоточного чтения запросов использовать Fixed thread pool
  \item Для многопотчной обработки полученного запроса использовать создание нового потока (java.lang.Thread)
  \item Для многопоточной отправки ответа использовать создание нового потока (java.lang.Thread)
  \item Для синхронизации доступа к коллекции использовать синхронизацию чтения и записи с помощью java.util.concurrent.locks.ReentrantLock
\end{itemize}
