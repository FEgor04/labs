\section{Текст задания}

Реализовать консольное приложение, которое реализует управление коллекцией объектов в интерактивном режиме.
В коллекции необходимо хранить объекты класса Vehicle, описание которого приведено ниже.

\textbf{Разработанная программа должна удовлетворять следующим требованиям:}

\begin{itemize}
\item Класс, коллекцией экземпляров которого управляет программа, должен реализовывать сортировку по умолчанию.
\item Все требования к полям класса (указанные в виде комментариев) должны быть выполнены.
\item Для хранения необходимо использовать коллекцию типа \texttt{java.util.TreeMap}
\item При запуске приложения коллекция должна автоматически заполняться значениями из файла.
\item Имя файла должно передаваться программе с помощью: \texttt{переменная окружения}.
\item Данные должны храниться в файле в формате \texttt{csv}
\item Чтение данных из файла необходимо реализовать с помощью класса \texttt{java.io.Input\-StreamReader}
\item Запись данных в файл необходимо реализовать с помощью класса \texttt{java.io.Output\-StreamWriter}
\item Все классы в программе должны быть задокументированы в формате \texttt{javadoc}.
\item Программа должна корректно работать с неправильными данными (ошибки пользовательского ввода, отсутсвие прав доступа к файлу и т.п.).
\end{itemize}

\textbf{В интерактивном режиме программа должна поддерживать выполнение следующих команд:}

\begin{itemize}
\item \texttt{help}: вывести справку по доступным командам
\item \texttt{info}: вывести в стандартный поток вывода информацию о коллекции
(тип, дата инициализации, количество элементов и т.д.)
\item \texttt{show}: вывести в стандартный поток вывода все элементы коллекции в строковом представлении
\item \texttt{insert null \{element\}}: добавить новый элемент с заданным ключом
\item \texttt{update id \{element\}}: обновить значение элемента коллекции, \texttt{id} которого
равен заданному
\item \texttt{remove\_key id}: удалить элемент коллекции по его ключу
\item \texttt{clear}: очистить коллекцию
\item \texttt{save}: сохранить коллекцию в файл
\item \texttt{execute\_script file\_name}: считать и исполнить скрипт из указанного файла.
В скрипте содержатся команды в таком же виде, в котором их вводит пользователь в интерактивном режиме.
\item \texttt{exit}: завершить программу (без сохранения в файл)
\item \texttt{remove\_greater \{element\}}: удалить из коллекции все элементы, превышающие заданный
\item \texttt{remove\_lower \{element\}}: удалить из коллекции все элементы, меньшие, чем заданный
\item \texttt{replace\_if\_lower null \{element\}}: заменить значение по ключу, если новое значение меньше старого
\item \texttt{min\_by\_id}: вывести любой объект из коллекции, значение поля \texttt{id}
которого является минимальным
\item \texttt{count\_by\_type type}: вывести количество элементов, значение поля \texttt{type}
которых равно заданному
\item \texttt{count\_less\_than\_engine\_power enginePower}: вывести количество элементов,
значение поля \texttt{enginePower} которых меньше заданного
\end{itemize}


\textbf{Формат ввода команд:}


\begin{itemize}
\item Все аргументы команды, являющиеся стандартными типами данных
(примитивные типы, классы-оболочки, \texttt{String}, классы для хранения дат),
должны вводиться в той же строке, что и имя команды.
\item Все составные типы данных (объекты классов, хранящиеся в коллекции)
должны вводиться по одному полю в строку.
\item При вводе составных типов данных пользователю должно показываться приглашение к вводу,
содержащее имя поля (например, "Введите дату рождения:")
\item Если поле является \texttt{enum}'ом, то вводится имя одной из его констант 
(при этом список констант должен быть предварительно выведен).
\item При некорректном пользовательском вводе
(введена строка, не являющаяся именем константы в \texttt{enum}'е; введена строка вместо числа;
введённое число не входит в указанные границы и т.п.) должно быть показано сообщение об ошибке и
предложено повторить ввод поля.
\item Для ввода значений \texttt{null} использовать пустую строку.
\item Поля с комментарием ``Значение этого поля должно генерироваться автоматически''
 не должны вводиться пользователем вручную при добавлении.
\end{itemize}

\textbf{Описание хранимых в коллекции классов:}

\begin{minted}{Java}
    public class Vehicle {
    private Integer id; //Поле не может быть null, Значение поля должно быть больше 0,
    // Значение этого поля должно быть уникальным, Значение этого поля должно
    // генерироваться автоматически
    private String name; //Поле не может быть null, Строка не может быть пустой
    private Coordinates coordinates; //Поле не может быть null
    private java.time.LocalDate creationDate; //Поле не может быть null,
    // Значение этого поля должно генерироваться автоматически
    private double enginePower; //Значение поля должно быть больше 0
    private VehicleType type; //Поле может быть null
    private FuelType fuelType; //Поле не может быть null
}
public class Coordinates {
    private Integer x; //Значение поля должно быть больше -523,
    // Поле не может быть null
    private long y;
}
public enum VehicleType {
    PLANE,
    SUBMARINE,
    BOAT,
    BICYCLE;
}
public enum FuelType {
    GASOLINE,
    ELECTRICITY,
    MANPOWER,
    PLASMA,
    ANTIMATTER;
}
\end{minted}
