\section{Текст задания}
\begin{enumerate}
\item Создать одномерный массив $a$ типа long. Заполнить его нечётными числами от $5$ до $15$ включительно в порядке возрастания.
\item Создать одномерный массив $x$ типа float. Заполнить его 12-ю случайными числами в диапазоне от $-6.0$ до $6.0$.
\item Создать двумерный массив $d$ размером $6 \times 12 $. Вычислить его элементы по следующей формуле (где $x = x[j]$):
\begin{itemize}
\item если $a[i] = 15$, то $d[i][j]=e^{e^{(2x)^3}}$;
\item если $a[i] \in \{5, 9, 13\}$, то $d[i][j]={(3 \cdot tan(\sqrt[3]{x}))}^2$;
\item для остальных значений $a[i]$: $d[i][j]=\sin(\arcsin(\frac{\pi}{4}\cdot e^{- | x |})$).
\end{itemize}
\item Напечатать полученный в результате массив в формате с четырьмя знаками после запятой.
\end{enumerate}