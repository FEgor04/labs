\section{Текст задания}
Разделить программу из лабораторной работы №5 на клиентский и серверный модули.
Серверный модуль должен осуществлять выполнение команд по управлению коллекцией. 
Клиентский модуль должен в интерактивном режиме считывать команды,
передавать их для выполнения на сервер и выводить результаты выполнения.

\textbf{Необходимо выполнить следующие требования:}
\begin{itemize}
\item Операции обработки объектов коллекции должны быть реализованы с помощью Stream API с использованием лямбда-выражений.
\item Объекты между клиентом и сервером должны передаваться в сериализованном виде.
\item Объекты в коллекции, передаваемой клиенту, должны быть отсортированы по умолчанию
\item Клиент должен корректно обрабатывать временную недоступность сервера.
\item Обмен данными между клиентом и сервером должен осуществляться по протоколу UDP
\item Для обмена данными на сервере необходимо использовать датаграммы
\item Для обмена данными на клиенте необходимо использовать сетевой канал
\item Сетевые каналы должны использоваться в неблокирующем режиме.
\end{itemize}

\textbf{Обязанности серверного приложения:}

\begin{itemize}
\item Работа с файлом, хранящим коллекцию.
\item Управление коллекцией объектов.
\item Назначение автоматически генерируемых полей объектов в коллекции.
\item Ожидание подключений и запросов от клиента.
\item Обработка полученных запросов (команд).
\item Сохранение коллекции в файл при завершении работы приложения.
\item Сохранение коллекции в файл при исполнении специальной команды, доступной только серверу (клиент такую команду отправить не может).
\item Серверное приложение должно состоять из следующих модулей (реализованных в виде одного или нескольких классов):
\item Модуль приёма подключений.
\item Модуль чтения запроса.
\item Модуль обработки полученных команд.
\item Модуль отправки ответов клиенту.
\item Сервер должен работать в однопоточном режиме.
\end{itemize}

\textbf{Обязанности клиентского приложения:}
\begin{itemize}
\item Чтение команд из консоли.
\item Валидация вводимых данных.
\item Сериализация введённой команды и её аргументов.
\item Отправка полученной команды и её аргументов на сервер.
\item Обработка ответа от сервера (вывод результата исполнения команды в консоль).
\item Команду save из клиентского приложения необходимо убрать.
\item Команда exit завершает работу клиентского приложения.
\end{itemize}

\textbf{Важно!}
Команды и их аргументы должны представлять из себя объекты классов.
Недопустим обмен "простыми" строками.
Так, для команды add или её аналога необходимо сформировать объект,
содержащий тип команды и объект, который должен храниться в вашей коллекции.

\textbf{Дополнительное задание:}
Реализовать логирование различных этапов работы сервера
(начало работы, получение нового подключения, получение нового запроса, отправка ответа и т.п.) с помощью \textbf{Log4J2}
