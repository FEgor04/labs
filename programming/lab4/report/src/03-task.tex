\section{Текст задания}
Доработать программу из лабораторной работы \textnumero 3, обновив реализацию объектной модели
в соответствии с новой версией описания предметной области.
\begin{quote}
Приняв такую позу, Незнайка стал делать попытки заснуть. Некоторое время он прислушивался к плавному шуму реактивного двигателя. Ему казалось, что двигатель потихоньку шепчет ему на ухо: "Чаф-чафчаф-чаф!" Эти звуки постепенно убаюкали Незнайку, и он заснул. Прошло несколько часов, и Незнайка почувствовал, что его кто-то тормошит за плечо. Открыв глаза, он увидел Пончика. Друзья поднялись в астрономическую кабину и взглянули в верхний иллюминатор. То, что они увидели, ошеломило их. Огромный светящийся шар висел над ракетой, заслоняя небо со звездами. Пончик напугался до того, что у него затряслись и губы, и щеки, и даже уши, а из глаз потекли слезы. Незнайка поднялся под потолок кабины и, прильнув к верхнему иллюминатору, принялся разглядывать поверхность Луны. Теперь Луна была видна так, как бывает видна в телескоп с Земли, и даже лучше. На ее поверхности вполне хорошо можно было разглядеть и горные цепи, и лунные цирки, и глубокие трещины или разломы. Пончик нехотя поднялся кверху и стал исподлобья поглядывать в иллюминатор. То, что он увидел, не принесло ему облегчения. Он заметил, что Луна теперь не стояла на месте, а приближалась с заметной скоростью. Сначала она была видна как огромный, величиной с полнеба, сверкающий круг. Мало-помалу этот круг разрастался и в конце концов заполнил собой все небо. Теперь, куда ни глянь, во все стороны простиралась поверхность Луны с опрокинутыми вверх ногами горными цепями, лунными кратерами и долинами. Все это угрожающе висело над головой и было уже так близко, что казалось, стоит только протянуть руку, и можно потрогать верхушку какой-нибудь лунной горы. Пончик боязливо поежился и, оттолкнувшись рукой от иллюминатора, опустился на дно кабины. Незнайка и сам не знал, как произойдет посадка на Луну, но ему хотелось показать Пончику, будто он все хорошо знает. Поэтому он сказал:
\end{quote}

Программа должна удовлетворять следующим требованиям:

\begin{itemize}
    \item В программе должны быть реализованы 2 собственных класса исключений (checked и unchecked), а также обработка исключений этих классов.
    \item В программу необходимо добавить использование локальных, анонимных и вложенных классов (static и non-static).
\end{itemize}
