\section{Текст задания}
\begin{quote}
Незнайка поднялся под потолок кабины и, прильнув к верхнему иллюминатору, принялся разглядывать поверхность Луны. Теперь Луна была видна так, как бывает видна в телескоп с Земли, и даже лучше. На ее поверхности вполне хорошо можно было разглядеть и горные цепи, и лунные цирки, и глубокие трещины или разломы. Пончик нехотя поднялся кверху и стал исподлобья поглядывать в иллюминатор. То, что он увидел, не принесло ему облегчения. Он заметил, что Луна теперь не стояла на месте, а приближалась с заметной скоростью. Сначала она была видна как огромный, величиной с полнеба, сверкающий круг. Мало-помалу этот круг разрастался и в конце концов заполнил собой все небо. Теперь, куда ни глянь, во все стороны простиралась поверхность Луны с опрокинутыми вверх ногами горными цепями, лунными кратерами и долинами. Все это угрожающе висело над головой и было уже так близко, что казалось, стоит только протянуть руку, и можно потрогать верхушку какой-нибудь лунной горы.
\end{quote}

Программа должна удовлетворять следующим требованиям:

\begin{itemize}
\item Доработанная модель должна соответствовать принципам SOLID.
\item Программа должна содержать как минимум два интерфейса и один абстрактный класс (номенклатура должна быть согласована с преподавателем).
\item В разработанных классах должны быть переопределены методы equals(), toString() и hashCode().
\item Программа должна содержать как минимум один перечисляемый тип (enum).
\end{itemize}

Порядок выполнения работы:
\begin{itemize}
\item Доработать объектную модель приложения.
\item Перерисовать диаграмму классов в соответствии с внесёнными в модель изменениями.
\item Согласовать с преподавателем изменения, внесённые в модель.
\item Модифицировать программу в соответствии с внесёнными в модель изменениями.
\end{itemize}