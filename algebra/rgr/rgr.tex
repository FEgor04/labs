\documentclass{article}
\usepackage[T2A]{fontenc}
\usepackage{amsmath} 
\usepackage[utf8]{inputenc}
\usepackage{hyperref}
\usepackage[english,russian]{babel}
\everymath{\displaystyle}
\usepackage[usenames]{color}
\usepackage{graphicx}

\usepackage{epigraph}

\usepackage{geometry}
\geometry{
  a4paper,
  top=25mm, 
  right=2cm,
  bottom=25mm, 
  left=2cm
}

\newcommand{\lin}{Lin}

\usepackage{fancyhdr}
\pagestyle{fancy}
\fancyhead{}
\fancyfoot[C]{\thepage}

\renewcommand{\headrulewidth}{0pt}
\usepackage{tikzsymbols}

\newcommand\lword[1]{\leavevmode\nobreak\hskip0pt plus\linewidth\penalty50\hskip0pt plus-\linewidth\nobreak\textbf{#1}}
\title{Поиск ЖНФ и жорданова базиса}
\author{Федоров Егор, P3115, вариант 44}
\date{}

\begin{document}
  \maketitle
  \epigraph{
    Невозможно объяснить, что такое \texttt{Матрица}. Ты должен увидеть это сам.
  }{Морфеус, <<Матрица>>}

  Исходная матрица:
  \[
  A = 
    \begin{pmatrix}
  2 & 0 & 0 & 0 & 0 & 0 \\
  0 & 2 & 0 & 0 & 0 & 0 \\
  2 & 0 & 2 & 0 & 0 & 0 \\
  0 & 3 & 0 & 2 & 0 & 0 \\
  1 & 0 & 0 & 3 & 2 & 0 \\
  0 & 2 & 1 & 0 & 0 & 2 \\
      \end{pmatrix}
  \]

  Найдем характеристический многочлен матрицы $\chi_\varphi(t) = A - E t$.
  Так как выше диагонали матрицы находятся одни нули, то определитель матрицы будет
  равен произведению чисел на диагонали. Таким образом, $\chi_\varphi(t) = (2-t)^6$.
  Значит оператор имеет единственное собственное число $2$ кратности $6$.

  Тогда $B = A - 2 E = \begin{pmatrix}
  0 & 0 & 0 & 0 & 0 & 0 \\
  0 & 0 & 0 & 0 & 0 & 0 \\
  2 & 0 & 0 & 0 & 0 & 0 \\
  0 & 3 & 0 & 0 & 0 & 0 \\
  1 & 0 & 0 & 3 & 0 & 0 \\
  0 & 2 & 1 & 0 & 0 & 0 \\
  \end{pmatrix}$.
  Приведем матрицу $B$ к ступенчатому виду.
  \[
    B = \begin{pmatrix}
  0 & 0 & 0 & 0 & 0 & 0 \\
  0 & 0 & 0 & 0 & 0 & 0 \\
  2 & 0 & 0 & 0 & 0 & 0 \\
  0 & 3 & 0 & 0 & 0 & 0 \\
  1 & 0 & 0 & 3 & 0 & 0 \\
  0 & 2 & 1 & 0 & 0 & 0 \\
  \end{pmatrix}
  \sim
  \begin{pmatrix}
  2 & 0 & 0 & 0 & 0 & 0 \\
  0 & 3 & 0 & 0 & 0 & 0 \\
  0 & 2 & 1 & 0 & 0 & 0 \\
  1 & 0 & 0 & 3 & 0 & 0 \\
  0 & 0 & 0 & 0 & 0 & 0 \\
  0 & 0 & 0 & 0 & 0 & 0 \\
  \end{pmatrix}
  \sim
  \begin{pmatrix}
  2 & 0 & 0 & 0 & 0 & 0 \\
  0 & 3 & 0 & 0 & 0 & 0 \\
  0 & 2 & 1 & 0 & 0 & 0 \\
  0 & 0 & 0 & 3 & 0 & 0 \\
  0 & 0 & 0 & 0 & 0 & 0 \\
  0 & 0 & 0 & 0 & 0 & 0 \\
  \end{pmatrix}
  \sim
  \begin{pmatrix}
  2 & 0 & 0 & 0 & 0 & 0 \\
  0 & 3 & 0 & 0 & 0 & 0 \\
  0 & 0 & 1 & 0 & 0 & 0 \\
  0 & 0 & 0 & 3 & 0 & 0 \\
  0 & 0 & 0 & 0 & 0 & 0 \\
  0 & 0 & 0 & 0 & 0 & 0 \\
  \end{pmatrix}
  \]

  Очевидно, что $\ker(B) = \lin\left(\left(0, 0, 0, 0, 1, 0\right)^T, \left(0, 0, 0, 0, 0, 1\right)^T\right)$.

  Найдем $\ker(B^2)$.

  \[
    B ^2 = B \cdot B =
    \begin{pmatrix}
      0 & 0 & 0 & 0 & 0 & 0 \\
      0 & 0 & 0 & 0 & 0 & 0 \\
      2 & 0 & 0 & 0 & 0 & 0 \\
      0 & 3 & 0 & 0 & 0 & 0 \\
      1 & 0 & 0 & 3 & 0 & 0 \\
      0 & 2 & 1 & 0 & 0 & 0 \\
    \end{pmatrix}
    \cdot
    \begin{pmatrix}
      0 & 0 & 0 & 0 & 0 & 0 \\
      0 & 0 & 0 & 0 & 0 & 0 \\
      2 & 0 & 0 & 0 & 0 & 0 \\
      0 & 3 & 0 & 0 & 0 & 0 \\
      1 & 0 & 0 & 3 & 0 & 0 \\
      0 & 2 & 1 & 0 & 0 & 0 \\
    \end{pmatrix}
    =
    \begin{pmatrix}
      0 & 0 & 0 & 0 & 0 & 0 \\
      0 & 0 & 0 & 0 & 0 & 0 \\
      0 & 0 & 0 & 0 & 0 & 0 \\
      0 & 0 & 0 & 0 & 0 & 0 \\
      0 & 9 & 0 & 0 & 0 & 0 \\
      2 & 0 & 0 & 0 & 0 & 0 \\
    \end{pmatrix}
    \sim
    \begin{pmatrix}
      2 & 0 & 0 & 0 & 0 & 0 \\
      0 & 9 & 0 & 0 & 0 & 0 \\
      0 & 0 & 0 & 0 & 0 & 0 \\
      0 & 0 & 0 & 0 & 0 & 0 \\
      0 & 0 & 0 & 0 & 0 & 0 \\
      0 & 0 & 0 & 0 & 0 & 0 \\
    \end{pmatrix}
  \]





\end{document}
