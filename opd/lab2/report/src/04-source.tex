\section{Текст исходной программы}
\begin{table}[h!]
    \centering
    \begin{longtable}{| c | c | c | p{9.5cm} |}
        \hline
        \small
        Адрес & Код команды & Мнемоника & Комментарии                                                                \\
        \hline
        142   & 0100        & X         & Хранение переменной X                                                      \\
        143   & A149        & A         & Хранение переменной A                                                      \\
        144   & 0200        & Y         & Хранение переменной Y                                                      \\
        145   & 0100        & B         & Хранение переменной B                                                      \\
        146   & 414A        & C         & Хранение переменной C                                                      \\
        147   & 0200        & D         & Хранение переменной D                                                      \\
        148   & A149        & E         & Хранение переменной E                                                      \\
        149   & E14A        & F         & Хранение переменной F                                                      \\
        14A   & E14A        & G         & Хранение переменной G                                                      \\
        14B   & 414A        & H         & Хранение переменной H                                                      \\
        \hline
        14C   & +A143       & LD \$A    & Записать значение переменной A в аккумулятор                               \\ % AC = A
        14D   & 3167        & OR \$K    & Записать в аккумулятор результат побитового логического или с переменной K \\ % AC = A | K
        14E   & E14A        & ST \$G    & Записать значение аккумулятора в переменную G                              \\ % G = A | K
        14F   & 0200        & CLA       & Очистить аккумулятор                                                       \\ % AC = 0
        150   & 4146        & ADD \$C   & Прибавить значение переменной C к аккумулятору                             \\ % AC = C
        151   & 414A        & ADD \$G   & Прибавить значение переменной G к аккумулятору                             \\ % AC = C + G = C + (A | K)
        152   & E14A        & ST \$G    & Записать значение аккумулятора в переменную G                              \\ % G = C + (A | K)
        153   & A147        & LD \$D    & Записать значение переменной D в аккумулятор                               \\ % AC = D
        154   & 314A        & OR \$G    & Записать в аккумулятор результат побитового логического или с переменной G \\ % AC = D | (C + (A | K))
        155   & E14A        & ST \$G    & Записать значение аккумулятора в переменную G                              \\ % G = D | (C + (A | K))
        156   & A144        & LD \$Y    & Записать значение переменной Y в аккумулятор                               \\ % AC = Y
        157   & 614A        & SUB \$G   & Вычесть из аккумулятора значение переменной G                              \\ % AC = Y - (D | (C + (A | K)))
        158   & E14A        & ST \$G    & Записать значение аккумулятора в переменную G                              \\ % G = Y - (D | (C + (A | K)))
        159   & A149        & LD \$F    & Записать значение переменной F в аккумулятор                               \\ % AC = F
        15A   & 214A        & AND \$G   & Записать в аккумулятор результат побитового логического И с переменной G   \\ % AC = F & (Y - (D | (C + (A | K))))
        15B   & E14A        & ST \$G    & Записать значение аккумулятора в переменную G                              \\ % G = F & (Y - (D | (C + (A | K))))
        15C   & A145        & LD \$B    & Записать значение переменной B в аккумулятор                               \\ % AC = B
        15D   & 414A        & ADD \$G   & Прибавить к аккумулятору значение переменной G                             \\ % AC = B + (F & (Y - (D | (C + (A | K)))))
        15E   & E14A        & ST \$G    & Записать значение аккумулятора в переменную G                              \\ %  G = B + (F & (Y - (D | (C + (A | K)))))
        15F   & 0200        & CLA       & Очистить аккумулятор                                                \\
        160   & 314B        & OR \$H    & Записать в аккумулятор побитовое ИЛИ с переменной H               \\ % AC = (0 | H) = H
        161   & 214A        & AND \$G   & Записать в аккумулятор результат побитового И аккумулятора и переменной G  \\ % AC = H & (B + (F & (Y - (D | (C + (A | K))))))
        162   & E14A        & ST \$G    & Записать значение аккумулятора в переменную G                              \\ % G = H & (B + (F & (Y - (D | (C + (A | K))))))
        163   & A148        & LD \$E    & Записать значение переменной E в аккумулятор                               \\ % AC = E
        164   & 414A        & ADD \$G   & Прибавить к аккумулятору значение переменной E                             \\ % AC = E + (H & (B + (F & (Y - (D | (C + (A | K)))))))
        165   & E142        & ST \$X    & Записать значение аккумулятора в переменную X                              \\ % X = E + (H & (B + (F & (Y - (D | (C + (A | K)))))))
        166   & 0100        & HLT       & Останов                                                                    \\
        \hline
        167   & E14A        & K         & Хранение переменной K                                                      \\
        \hline
    \end{longtable}
    \caption{Текст исходной программы}
\end{table}

Реализуемая формула: $X = E + (H \& (B + (F \& (Y - (D | (C + (A | K)))))))$
\begin{itemize}
    \item $A$ --- набор из 16 логических значений
    \item $B$ --- 16-разрядное знаковое число
    \item $C$ --- 16-разрядное знаковое число
    \item $D$ --- набор из 16 логических значений
    \item $E$ --- 16-разрядное знаковое число
    \item $F$ --- набор из 16 логических значений
    \item $G$ --- 16-разрядное знаковое число
    \item $H$ --- набор из 16 логических значений
\end{itemize}


\clearpage
\section{Вариант с меньшим количеством команд}
\lstinputlisting[language={[x86masm]Assembler}]{less_commands.asm}

\clearpage
\section{Трассировка}
\begin{table}[ht]
    \centering
    \resizebox{\textwidth}{!}{%
    \begin{tabular}{| c | c | c | c | c | c | c | c | c | c | c | c |}
        \hline
        \multicolumn{2}{|p{4cm}|}{Выполняемая команда} &
        \multicolumn{8}{|p{8cm}|}{Содержимое регистров процессора после выполнения команды} &
        \multicolumn{2}{p{3cm}|}{Ячейка, содержимое которой изменилось после выполнения команды} \\
        \hline
        Адрес & Код & IP & CR & AR & DR & SP & BR & AC & NZVC & Адрес & Новый код \\ \hline
        14C & A143 & 14C & 0000 & 000 & 0000 & 000 & 0000 & 0000 & 0100 & ~ & ~ \\ \hline
        14C & A143 & 14D & A143 & 143 & A149 & 000 & 014C & A149 & 1000 & ~ & ~ \\ \hline
        14D & 3167 & 14E & 3167 & 167 & E14A & 000 & 1EB4 & E14B & 1000 & ~ & ~ \\ \hline
        14E & E14A & 14F & E14A & 14A & E14B & 000 & 014E & E14B & 1000 & 14A & E14B \\ \hline
        14F & 0200 & 150 & 0200 & 14F & 0200 & 000 & 014F & 0000 & 0100 & ~ & ~ \\ \hline
        150 & 4146 & 151 & 4146 & 146 & 414A & 000 & 0150 & 414A & 0000 & ~ & ~ \\ \hline
        151 & 414A & 152 & 414A & 14A & E14B & 000 & 0151 & 2295 & 0001 & ~ & ~ \\ \hline
        152 & E14A & 153 & E14A & 14A & 2295 & 000 & 0152 & 2295 & 0001 & 14A & 2295 \\ \hline
        153 & A147 & 154 & A147 & 147 & 0200 & 000 & 0153 & 0200 & 0001 & ~ & ~ \\ \hline
        154 & 314A & 155 & 314A & 14A & 2295 & 000 & DD6A & 2295 & 0001 & ~ & ~ \\ \hline
        155 & E14A & 156 & E14A & 14A & 2295 & 000 & 0155 & 2295 & 0001 & 14A & 2295 \\ \hline
        156 & A144 & 157 & A144 & 144 & 0200 & 000 & 0156 & 0200 & 0001 & ~ & ~ \\ \hline
        157 & 614A & 158 & 614A & 14A & 2295 & 000 & 0157 & DF6B & 1000 & ~ & ~ \\ \hline
        158 & E14A & 159 & E14A & 14A & DF6B & 000 & 0158 & DF6B & 1000 & 14A & DF6B \\ \hline
        159 & A149 & 15A & A149 & 149 & E14A & 000 & 0159 & E14A & 1000 & ~ & ~ \\ \hline
        15A & 214A & 15B & 214A & 14A & DF6B & 000 & 015A & C14A & 1000 & ~ & ~ \\ \hline
        15B & E14A & 15C & E14A & 14A & C14A & 000 & 015B & C14A & 1000 & 14A & C14A \\ \hline
        15C & A145 & 15D & A145 & 145 & 0100 & 000 & 015C & 0100 & 0000 & ~ & ~ \\ \hline
        15D & 414A & 15E & 414A & 14A & C14A & 000 & 015D & C24A & 1000 & ~ & ~ \\ \hline
        15E & E14A & 15F & E14A & 14A & C24A & 000 & 015E & C24A & 1000 & 14A & C24A \\ \hline
        15F & 0200 & 160 & 0200 & 15F & 0200 & 000 & 015F & 0000 & 0100 & ~ & ~ \\ \hline
        160 & 314B & 161 & 314B & 14B & 414A & 000 & BEB5 & 414A & 0000 & ~ & ~ \\ \hline
        161 & 214A & 162 & 214A & 14A & C24A & 000 & 0161 & 404A & 0000 & ~ & ~ \\ \hline
        162 & E14A & 163 & E14A & 14A & 404A & 000 & 0162 & 404A & 0000 & 14A & 404A \\ \hline
        163 & A148 & 164 & A148 & 148 & A149 & 000 & 0163 & A149 & 1000 & ~ & ~ \\ \hline
        164 & 414A & 165 & 414A & 14A & 404A & 000 & 0164 & E193 & 1000 & ~ & ~ \\ \hline
        165 & E142 & 166 & E142 & 142 & E193 & 000 & 0165 & E193 & 1000 & 142 & E193 \\ \hline
        166 & 0100 & 167 & 0100 & 166 & 0100 & 000 & 0166 & E193 & 1000 & ~ & ~ \\ \hline

    \end{tabular}}
\end{table}
