\section{Текст исходной программы}
\begin{table}[h!]
    \centering
    \begin{longtable}{| c | c | c | p{9.5cm} |}
        \hline
        \small
        Адрес & Код команды & Мнемоника           & Комментарии                                                                      \\
        \hline
        \hline
        444   & 0457        & A                   & Указатель на первый элемент массива                                      \\
        445   & 0200        & B                   & Указатель на текущий элемент массива, станет = 457 после команды 44D      \\
        446   & 4000        & C                   & Количество элементов массива, станет = 5 после команды 44B               \\
        447   & 0200        & D                   & Результат --- количество нечетных элементов \\
        \hline
        \hline
        448   & +0200       & CLA                 & Очистка аккумулятора                                                             \\
        449   & EEFD        & ST D                & Прямая относительная адресация                                                    \\ % D = 0
        44A   & AF05        & LD \#05             & Прямая загрузка операнда                                                         \\ % AC = 5
        44B   & EEFA        & ST C                & Прямая относительная адресация                                                    \\ % C = 5
        44C   & AEF7        & LD A                & Прямая относительная адресация                                                    \\ % AC=A=0457
        44D   & EEF7        & ST B                & Прямая относительная адресация                                                    \\ % B = 0457
        % A = 0457; B = 0457; C = 5; D = 0
        \hline
        44E   & AAF6        & LD (B)+             & Косвенная автоинкрементная, загружает первый элемент массива                     \\
        44F   & 0480        & ROR                 & Сдвиг вправо, $C=Arr_i \pmod 2, AC = \left\lfloor \frac{Arr_i}{2} \right\rfloor$ \\
        450   & F401        & BCS (IP+1)+1        & Переход к 452 если перенос $\Leftrightarrow Arr_i \pmod 2 = 1$, иначе \ldots                   \\
        451   & CE02        & JUMP (IP+1)+2=454   & переход к адресу 454                                                             \\
        452   & 0400        & ROL                 & Сдвиг влево, $AC = Arr_i$                                                        \\
        453   & 6AF3        & SUB (D)+            & Косвенная автоинкрементая, будет вычитать нули, при этом увеличивая D на 1.                                   \\
        454   & 8446        & LOOP \$C            & ---                                                                              \\ % Если C > 0, то ...
        455   & CEF8        & JUMP (IP+1)-8 = 44E & Прямая относительная                                                             \\ % Переход на 44E
        456   & 0100        & HLT                 & ---                                                                              \\ % Иначе останов
        \hline
        \hline
        457   & 0C00        & ---                 & $Arr_1$                                                                          \\
        458   & 0501        & ---                 & $Arr_2$                                                                          \\
        459   & 0200        & ---                 & $Arr_3$                                                                          \\
        45A   & D455        & ---                 & $Arr_4$                                                                          \\
        45B   & 1000        & ---                 & $Arr_5$                                                                          \\
        \hline
    \end{longtable}
    \caption{Текст исходной программы}
\end{table}

\subsection{Предназначение и описание программы}
Программа считает количество нечетных элементов в массиве $Arr$.
\begin{itemize}
    \item 445 --- 447: исходные данные
    \item 448 --- 44D: установка изначальных данных программы
    \item 44E --- 456: итерация по всем элементам массива, проверка на нечетность, инкремент счетчика
    \item 457 --- 45B: массив
\end{itemize}

\subsection{Область представления}
\begin{itemize}
    \item $A$ --- указатель на начало массива, 11-битное беззнаковое число (адрес) % (11 бит так как в БЭВМ всего 2048)
    \item $C$ --- количество элементов в массиве, 11-битное беззнаковое число
    \item $D$ --- результат, 16-битное беззнаковое число
    \item $Arr$ --- исходный массив
\end{itemize}

\subsection{Область допустимых значений}
Для всех случаев:
\[ \forall i \in \left\{0, 1, \ldots, C-1  \right\} \quad -2^{15} \leq Arr_i \leq 2^{15} - 1 \]
% \[ 1 \leq C \leq 127 \]
% \[ (\underbrace{\{D, D+1, \ldots, D+C-1\}}_{\text{элементы (D)+}}) \cap (\underbrace{\{A, A+1, \ldots, A+C-1\}}_{\text{массив}} \cup \underbrace{\{444, 445, \ldots, 456\}}_{\text{программа}}) = \emptyset \]
\[ D = 0 \]
\begin{itemize}

    \item Случай 1. Массив до программы:
    $0 \leq A \leq 444_{16} - C$, $1 \leq C \leq 444_{16} = 1092_{10}$
    \[ \begin{cases}
        0 \leq A \leq 444_{16} - C \\
        1 \leq C \leq 1092
    \end{cases} \]

    \item Случай 2. Массив после программы:
    $457 \leq A \leq 7FF_{16} - C$, $1 \leq C \leq 7FF_{16} - 457_{16} + 1 = 3A8_{16} = 936$
    \[ \begin{cases}
        457 \leq A \leq 7FF_{16} - C \\
        1 \leq C \leq 936
    \end{cases} \]
\end{itemize}


\clearpage
\subsection{Трассировка}
\begin{table}[!ht]
    \centering
    \resizebox{!}{0.45\textheight}{
    \begin{tabular}{|l|l|l|l|l|l|l|l|l|l|l|l|}
    \hline
        Адр & Знчн & IP & CR & AR & DR & SP & BR & AC & NZVC & Адр & Знчн \\ \hline
        444 & 0457 & 444 & 0000 & 000 & 0000 & 000 & 0000 & 0000 & 0100 & ~ & ~ \\ \hline
        444 & 0457 & 445 & 0457 & 444 & 0457 & 000 & 0444 & 0000 & 0100 & ~ & ~ \\ \hline
        445 & 0200 & 446 & 0200 & 445 & 0200 & 000 & 0445 & 0000 & 0100 & ~ & ~ \\ \hline
        446 & 4000 & 447 & 4000 & 000 & 0000 & 000 & 0446 & 0000 & 0100 & ~ & ~ \\ \hline
        447 & 0200 & 448 & 0200 & 447 & 0200 & 000 & 0447 & 0000 & 0100 & ~ & ~ \\ \hline
        448 & 0200 & 449 & 0200 & 448 & 0200 & 000 & 0448 & 0000 & 0100 & ~ & ~ \\ \hline
        449 & EEFD & 44A & EEFD & 447 & 0000 & 000 & FFFD & 0000 & 0100 & 447 & 0000 \\ \hline
        44A & AF05 & 44B & AF05 & 44A & 0005 & 000 & 0005 & 0005 & 0000 & ~ & ~ \\ \hline
        44B & EEFA & 44C & EEFA & 446 & 0005 & 000 & FFFA & 0005 & 0000 & 446 & 0005 \\ \hline
        44C & AEF7 & 44D & AEF7 & 444 & 0457 & 000 & FFF7 & 0457 & 0000 & ~ & ~ \\ \hline
        44D & EEF7 & 44E & EEF7 & 445 & 0457 & 000 & FFF7 & 0457 & 0000 & 445 & 0457 \\ \hline
        44E & AAF6 & 44F & AAF6 & 457 & 0C00 & 000 & FFF6 & 0C00 & 0000 & 445 & 0458 \\ \hline
        44F & 0480 & 450 & 0480 & 44F & 0480 & 000 & 044F & 0600 & 0000 & ~ & ~ \\ \hline
        450 & F501 & 452 & F501 & 450 & F501 & 000 & 0001 & 0600 & 0000 & ~ & ~ \\ \hline
        452 & 0400 & 453 & 0400 & 452 & 0400 & 000 & 0452 & 0C00 & 0000 & ~ & ~ \\ \hline
        453 & 6AF3 & 454 & 6AF3 & 000 & 0000 & 000 & FFF3 & 0C00 & 0001 & 447 & 0001 \\ \hline
        454 & 8446 & 455 & 8446 & 446 & 0004 & 000 & 0003 & 0C00 & 0001 & 446 & 0004 \\ \hline
        455 & CEF8 & 44E & CEF8 & 455 & 044E & 000 & FFF8 & 0C00 & 0001 & ~ & ~ \\ \hline
        44E & AAF6 & 44F & AAF6 & 458 & 0501 & 000 & FFF6 & 0501 & 0001 & 445 & 0459 \\ \hline
        44F & 0480 & 450 & 0480 & 44F & 0480 & 000 & 044F & 8280 & 1001 & ~ & ~ \\ \hline
        450 & F501 & 451 & F501 & 450 & F501 & 000 & 0450 & 8280 & 1001 & ~ & ~ \\ \hline
        451 & CE02 & 454 & CE02 & 451 & 0454 & 000 & 0002 & 8280 & 1001 & ~ & ~ \\ \hline
        454 & 8446 & 455 & 8446 & 446 & 0003 & 000 & 0002 & 8280 & 1001 & 446 & 0003 \\ \hline
        455 & CEF8 & 44E & CEF8 & 455 & 044E & 000 & FFF8 & 8280 & 1001 & ~ & ~ \\ \hline
        44E & AAF6 & 44F & AAF6 & 459 & 0200 & 000 & FFF6 & 0200 & 0001 & 445 & 045A \\ \hline
        44F & 0480 & 450 & 0480 & 44F & 0480 & 000 & 044F & 8100 & 1010 & ~ & ~ \\ \hline
        450 & F501 & 452 & F501 & 450 & F501 & 000 & 0001 & 8100 & 1010 & ~ & ~ \\ \hline
        452 & 0400 & 453 & 0400 & 452 & 0400 & 000 & 0452 & 0200 & 0011 & ~ & ~ \\ \hline
        453 & 6AF3 & 454 & 6AF3 & 001 & 0000 & 000 & FFF3 & 0200 & 0001 & 447 & 0002 \\ \hline
        454 & 8446 & 455 & 8446 & 446 & 0002 & 000 & 0001 & 0200 & 0001 & 446 & 0002 \\ \hline
        455 & CEF8 & 44E & CEF8 & 455 & 044E & 000 & FFF8 & 0200 & 0001 & ~ & ~ \\ \hline
        44E & AAF6 & 44F & AAF6 & 45A & D455 & 000 & FFF6 & D455 & 1001 & 445 & 045B \\ \hline
        44F & 0480 & 450 & 0480 & 44F & 0480 & 000 & 044F & EA2A & 1001 & ~ & ~ \\ \hline
        450 & F501 & 451 & F501 & 450 & F501 & 000 & 0450 & EA2A & 1001 & ~ & ~ \\ \hline
        451 & CE02 & 454 & CE02 & 451 & 0454 & 000 & 0002 & EA2A & 1001 & ~ & ~ \\ \hline
        454 & 8446 & 455 & 8446 & 446 & 0001 & 000 & 0000 & EA2A & 1001 & 446 & 0001 \\ \hline
        455 & CEF8 & 44E & CEF8 & 455 & 044E & 000 & FFF8 & EA2A & 1001 & ~ & ~ \\ \hline
        44E & AAF6 & 44F & AAF6 & 45B & 1000 & 000 & FFF6 & 1000 & 0001 & 445 & 045C \\ \hline
        44F & 0480 & 450 & 0480 & 44F & 0480 & 000 & 044F & 8800 & 1010 & ~ & ~ \\ \hline
        450 & F501 & 452 & F501 & 450 & F501 & 000 & 0001 & 8800 & 1010 & ~ & ~ \\ \hline
        452 & 0400 & 453 & 0400 & 452 & 0400 & 000 & 0452 & 1000 & 0011 & ~ & ~ \\ \hline
        453 & 6AF3 & 454 & 6AF3 & 002 & 0000 & 000 & FFF3 & 1000 & 0001 & 447 & 0003 \\ \hline
        454 & 8446 & 456 & 8446 & 446 & 0000 & 000 & FFFF & 1000 & 0001 & 446 & 0000 \\ \hline
        456 & 0100 & 457 & 0100 & 456 & 0100 & 000 & 0456 & 1000 & 0001 & ~ & ~ \\ \hline
    \end{tabular}}
\end{table}
