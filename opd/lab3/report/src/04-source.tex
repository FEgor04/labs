\section{Текст исходной программы}
\begin{table}[h!]
    \centering
    \begin{longtable}{| c | c | c | p{9.5cm} |}
        \hline
        \small
        Адрес & Код команды & Мнемоника           & Комментарии                                                                      \\
        \hline
        \hline
        444   & 0457        & A                   & Указатель на первый элемент массива                                      \\
        445   & 0200        & B                   & Указатель на текущий элемент массива, станет = 457 после команды 44D      \\
        446   & 4000        & C                   & Количество элементов массива, станет = 5 после команды 44B               \\
        447   & 0200        & D                   & Результат --- количество нечетных элементов \\
        \hline
        \hline
        448   & +0200       & CLA                 & Очистка аккумулятора                                                             \\
        449   & EEFD        & ST D                & Прямая относительная адресация                                                    \\ % D = 0
        44A   & AF05        & LD \#05             & Прямая загрузка операнда                                                         \\ % AC = 5
        44B   & EEFA        & ST C                & Прямая относительная адресация                                                    \\ % C = 5
        44C   & AEF7        & LD A                & Прямая относительная адресация                                                    \\ % AC=A=0457
        44D   & EEF7        & ST B                & Прямая относительная адресация                                                    \\ % B = 0457
        % A = 0457; B = 0457; C = 5; D = 0
        \hline
        44E   & AAF6        & LD (B)+             & Косвенная автоинкрементная, загружает первый элемент массива                     \\
        44F   & 0480        & ROR                 & Сдвиг вправо, $C=Arr_i \pmod 2, AC = \left\lfloor \frac{Arr_i}{2} \right\rfloor$ \\
        450   & F401        & BCS (IP+1)+1        & Переход к 452 если перенос $\Leftrightarrow Arr_i \pmod 2 = 1$, иначе \ldots                   \\
        451   & CE02        & JUMP (IP+1)+2=454   & переход к адресу 454                                                             \\
        452   & 0400        & ROL                 & Сдвиг влево, $AC = Arr_i$                                                        \\
        453   & 6AF3        & SUB (D)+            & Косвенная автоинкрементая, будет вычитать нули, при этом увеличивая D на 1.                                   \\
        454   & 8446        & LOOP \$C            & ---                                                                              \\ % Если C > 0, то ...
        455   & CEF8        & JUMP (IP+1)-8 = 44E & Прямая относительная                                                             \\ % Переход на 44E
        456   & 0100        & HLT                 & ---                                                                              \\ % Иначе останов
        \hline
        \hline
        457   & 0C00        & ---                 & $Arr_1$                                                                          \\
        458   & 0501        & ---                 & $Arr_2$                                                                          \\
        459   & 0200        & ---                 & $Arr_3$                                                                          \\
        45A   & D455        & ---                 & $Arr_4$                                                                          \\
        45B   & 1000        & ---                 & $Arr_5$                                                                          \\
        \hline
    \end{longtable}
    \caption{Текст исходной программы}
\end{table}

\subsection{Предназначение и описание программы}
Программа считает количество нечетных элементов в массиве $Arr$.
\begin{itemize}
    \item 445 --- 447: исходные данные
    \item 448 --- 44D: установка изначальных данных программы
    \item 44E --- 456: итерация по всем элементам массива, проверка на нечетность, инкремент счетчика
    \item 457 --- 45B: массив
\end{itemize}

\subsection{Область представления}
\begin{itemize}
    \item $A$ --- указатель на начало массива, 11-битное беззнаковое число
    \item $C$ --- количество элементов в массиве, 7-разрядное беззнаковое число
    \item $D$ --- результат, 16-битное беззнаковое число
    \item $Arr$ --- исходный массив
\end{itemize}

\subsection{Область допустимых значений}
Для всех случаев:
\[ \forall i \in \left\{0, 1, \ldots, C-1  \right\} \quad -2^{15} \leq Arr_i \leq 2^{15} - 1 \]
\[ 1 \leq C \leq 127 \]
\[ (\underbrace{\{D, D+1, \ldots, D+C-1\}}_{\text{элементы (D)+}}) \cap (\underbrace{\{A, A+1, \ldots, A+C-1\}}_{\text{массив}} \cup \underbrace{\{444, 445, \ldots, 456\}}_{\text{программа}}) = \emptyset \]
\begin{itemize}

    \item Случай 1. Массив до программы:
    $0 \leq A \leq 444 - C$

    \item Случай 2. Массив после программы:
    $456 < A \leq 7FF_{16}$
\end{itemize}