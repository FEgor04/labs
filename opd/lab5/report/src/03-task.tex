\section{Текст задания}
По выданному преподавателем варианту разработать программу асинхронного обмена данными с внешним устройством. 
При помощи программы осуществить ввод или вывод информации, используя в качестве подтверждения данных сигнал (кнопку) готовности ВУ.
\begin{enumerate}
	\item Программа осуществляет асинхронный ввод данных с ВУ-2
	\item Программа начинается с адреса $22C_{16}$. Размещаемая строка находится по адресу $618_{16}$.
	\item Строка должна быть представлена в кодировке КОИ-8.
	\item Формат представления строки в памяти: \texttt{АДР1: СИМВ1 СИМВ2 АДР2: СИМВ3 СИМВ4 ... СТОП\_СИМВ}.
	\item Ввод или вывод строки должен быть завершен по символу c кодом \texttt{00 (NUL)}. Стоп символ является обычным символом строки и подчиняется тем же правилам расположения в памяти что и другие символы строки.
\end{enumerate}
