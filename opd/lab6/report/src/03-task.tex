\section{Текст задания}
По выданному преподавателем варианту разработать и исследовать работу комплекса программ обмена данными в режиме прерывания программы.
Основная программа должна изменять содержимое заданной ячейки памяти (Х), которое должно быть представлено как знаковое число.
Область допустимых значений изменения Х должна быть ограничена заданной функцией F(X) и конструктивными особенностями регистра данных ВУ (8-ми битное знаковое представление).
Программа обработки прерывания должна выводить на ВУ модифицированное значение Х в соответствии с вариантом задания, а также игнорировать все необрабатываемые прерывания.

\begin{enumerate}
\item Основная программа должна уменьшать на 3 содержимое X (ячейки памяти с адресом 04816) в цикле.
\item Обработчик прерывания должен по нажатию кнопки готовности ВУ-3 осуществлять вывод результата вычисления функции F(X)=-5X+2 на данное ВУ,
a по нажатию кнопки готовности ВУ-2 вычесть содержимое РД данного ВУ из Х, результат записать в X
\item Если Х оказывается вне ОДЗ при выполнении любой операции по его изменению, то необходимо в Х записать максимальное по ОДЗ число.

\end{enumerate}

