\section{Текст исходной программы}
\begin{table}[h!]
    \centering
    \begin{longtable}{| c | c | c | p{9.5cm} |}
        \hline
        \small
        Адрес & Код команды & Мнемоника           & Комментарии                                           \\
        33D   & 0200        & CLA                 & ---                                                   \\
        33E   & EE19        & ST (IP+1)+25 = 358  & ST R, R = 0                                             \\
        33F   & AE15        & LD (IP+1)+21 = 355  & LD Z                                                   \\
        340   & 0C00        & PUSH                & Добавляет AC=355 в стэк                               \\
        341   & D702        & CALL 702            & Вызывает подпрограмму с аргументом равным 0x355       \\
        342   & 0800        & POP                 & Помещает вершину стэка (ячейку 355) в аккумулятор     \\
        343   & 0740        & DEC                 & AC = AC - 1 = F(Z) - 1                                          \\ 
        344   & 6E13        & SUB (IP+1)+19 = 358 & SUB 0 WTF                                             \\
        345   & EE12        & ST (IP+1)+18 = 358  & R = F(Z)-1                                            \\
        346   & AE10        & LD (IP+1)+16 = 357  & AC = X                                            \\
        347   & 0740        & DEC                 & AC = X - 1                                      \\
        348   & 0C00        & PUSH                & Добавляет AC в стэк                                   \\
        349   & D702        & CALL 702            & Вызывает подпрограмму с аргументом равным (0x357) - 1 \\
        34A   & 0800        & POP                 & AC = F(X-1)                                                   \\
        34B   & 6E0C        & SUB (IP+1)+12 = 358 & AC = F(X-1) - (F(Z)-1)                                                   \\
        34C & EE0B & ST (IP+1)+11 = 358 & R = F(X-1) - (F(Z)-1) \\
        34D & AE08 & LD (IP+1)+8 = 356 & LD Y \\
        34E & 0740 & DEC & AC = Y - 1 \\
        34F & 0C00 & PUSH & --- \\
        350 & D702 & CALL \$F & --- \\
        351 & 0800 & POP & AC = F(Y-1) \\
        352 & 4E05 & ADD (IP+1)+5 = 358 & ADD R, AC = F(Y-1) + F(X-1) - F(Z) + 1 \\
        353 & EE04 & ST (IP+1+4) = 358 & ST R \\
        354 & 0100 & HLT & --- \\
        355 & ZZZZ & Z & Переменная Z \\
        356 & YYYY & Y & Переменная Y \\
        357 & XXXX & X & Переменная X \\
        358 & FF8C & R & Результат \\
        \hline
        \hline
        702   & AC01        & LD \&1              & Загружает в аккумулятор аргумент функции              \\
        703   & F001        & BEQ (IP+1)+1 = 705  & Переход если аргумент равен 0                         \\
        704   & F307        & BPL (IP+1)+7 = 70C  & Переход если аргумент > 0                             \\
        705   & 7E09        & CMP (IP+1)+9 = 70E  & Установить значение флагов по операции AC - 70E       \\
        706   & F805        & BLT (IP+1)+5 = 70C  & Переход если строго меньше                            \\
        707   & F004        & BEQ (IP+1)+4 = 70C  & Переход если равно 0                                  \\
        708   & 0500        & ASL                 & Сдвиг влево, умножение на два                                                   \\
        709   & 0500        & ASL                 & Сдвиг влево, умножение на два                                                   \\
        70A   & 6E05        & SUB (IP+1)+5 = 710  & ---                                                   \\
        70B   & CE01        & JUMP (IP+1)+1 = 70D & ---                                                   \\
        70C   & AE02        & LD (IP+1)+2 = 70F   & ---                                                   \\
        70D   & EC01        & ST \&1              & ---                                                   \\
        70E   & 0A00        & RET                 & ---                                                   \\
        70F   & F8A5        & BLT (IP+1+165)=7B4  & ---                                                   \\
        710 & 006D & Переменная для сравнения & --- \\
        \hline
    \end{longtable}
    \caption{Текст исходной программы}
\end{table}

\subsection{Предназначение и описание программы}
Программа вычисляет значение формулы $F(X-1) + F(Y-1) - F(Z) + 1$ для данных $X, Y, Z$.

\[ 
   F(X) = \begin{cases}
    F8A5_{16}, & x > 0 \\
    4x - 006D_{16}, & 2560 < x < 0 \\
    F8A5_{16}, & x - 0A00_{16} < 0 \\
   \end{cases}
\]

Второе условие системы никогда не выполняется, поэтому функция всегда возвращает константу.
В десятичной системе:
\[ 
   F(X) = -1883
\]
\begin{figure}
    \centering
    \includesvg[width=\textwidth]{img/desmos-graph.svg}
    \caption{График функции, реализуемой подпрограммой}
\end{figure}

\subsection{Область представления}
Для программы:
$X, Y, Z, R$ --- 16-битные знаковые числа.

Для подпрограммы: $X$ --- 16 битное знаковое число.
\subsection{Область допустимых значений}

Для программы: $2^{15} \leq x \leq 2^{15} - 1$
Для подпрограммы: $2^{15} \leq x < 2^{15} - 1$ --- так как в любом случае функция возвращает константу.

\subsection{Трассировка}
\begin{table}[!ht]
    \centering
    \begin{tabular}{|l|l|l|l|l|l|l|l|l|l|l|l|}
    \hline
        \textbf{Адр} & \textbf{Знчн} & \textbf{IP} & \textbf{CR} & \textbf{AR} & \textbf{DR} & \textbf{SP} & \textbf{BR} & \textbf{AC} & \textbf{NZVC} & \textbf{Адр} & \textbf{Знчн} \\ \hline
        33D & 0200 & 33D & 0000 & 000 & 0000 & 000 & 0000 & 0000 & 0100 & ~ & ~ \\ \hline
        33D & 0200 & 33E & 0200 & 33D & 0200 & 000 & 033D & 0000 & 0100 & ~ & ~ \\ \hline
        33E & EE19 & 33F & EE19 & 358 & 0000 & 000 & 0019 & 0000 & 0100 & 358 & 0000 \\ \hline
        33F & AE15 & 340 & AE15 & 355 & 0000 & 000 & 0015 & 0000 & 0100 & ~ & ~ \\ \hline
        340 & 0C00 & 341 & 0C00 & 7FF & 0000 & 7FF & 0340 & 0000 & 0100 & 7FF & 0000 \\ \hline
        341 & D359 & 359 & D359 & 7FE & 0342 & 7FE & D359 & 0000 & 0100 & 7FE & 0342 \\ \hline
        359 & AC01 & 35A & AC01 & 7FF & 0000 & 7FE & 0001 & 0000 & 0100 & ~ & ~ \\ \hline
        35A & F001 & 35C & F001 & 35A & F001 & 7FE & 0001 & 0000 & 0100 & ~ & ~ \\ \hline
        35C & 7E07 & 35D & 7E07 & 364 & 0A00 & 7FE & 0007 & 0000 & 1000 & ~ & ~ \\ \hline
        35D & F804 & 362 & F804 & 35D & F804 & 7FE & 0004 & 0000 & 1000 & ~ & ~ \\ \hline
        362 & AE02 & 363 & AE02 & 365 & F8A5 & 7FE & 0002 & F8A5 & 1000 & ~ & ~ \\ \hline
        363 & EC01 & 364 & EC01 & 7FF & F8A5 & 7FE & 0001 & F8A5 & 1000 & 7FF & F8A5 \\ \hline
        364 & 0A00 & 342 & 0A00 & 7FE & 0342 & 7FF & 0364 & F8A5 & 1000 & ~ & ~ \\ \hline
        342 & 0800 & 343 & 0800 & 7FF & F8A5 & 000 & 0342 & F8A5 & 1000 & ~ & ~ \\ \hline
        343 & 0740 & 344 & 0740 & 343 & 0740 & 000 & 0343 & F8A4 & 1001 & ~ & ~ \\ \hline
        344 & 6E13 & 345 & 6E13 & 358 & 0000 & 000 & 0013 & F8A4 & 1001 & ~ & ~ \\ \hline
        345 & EE12 & 346 & EE12 & 358 & F8A4 & 000 & 0012 & F8A4 & 1001 & 358 & F8A4 \\ \hline
        346 & AE10 & 347 & AE10 & 357 & 0000 & 000 & 0010 & 0000 & 0101 & ~ & ~ \\ \hline
        347 & 0740 & 348 & 0740 & 347 & 0740 & 000 & 0347 & FFFF & 1000 & ~ & ~ \\ \hline
        348 & 0C00 & 349 & 0C00 & 7FF & FFFF & 7FF & 0348 & FFFF & 1000 & 7FF & FFFF \\ \hline
        349 & D359 & 359 & D359 & 7FE & 034A & 7FE & D359 & FFFF & 1000 & 7FE & 034A \\ \hline
        359 & AC01 & 35A & AC01 & 7FF & FFFF & 7FE & 0001 & FFFF & 1000 & ~ & ~ \\ \hline
        35A & F001 & 35B & F001 & 35A & F001 & 7FE & 035A & FFFF & 1000 & ~ & ~ \\ \hline
        35B & F306 & 35C & F306 & 35B & F306 & 7FE & 035B & FFFF & 1000 & ~ & ~ \\ \hline
        35C & 7E07 & 35D & 7E07 & 364 & 0A00 & 7FE & 0007 & FFFF & 1001 & ~ & ~ \\ \hline
        35D & F804 & 362 & F804 & 35D & F804 & 7FE & 0004 & FFFF & 1001 & ~ & ~ \\ \hline
        362 & AE02 & 363 & AE02 & 365 & F8A5 & 7FE & 0002 & F8A5 & 1001 & ~ & ~ \\ \hline
        363 & EC01 & 364 & EC01 & 7FF & F8A5 & 7FE & 0001 & F8A5 & 1001 & 7FF & F8A5 \\ \hline
        364 & 0A00 & 34A & 0A00 & 7FE & 034A & 7FF & 0364 & F8A5 & 1001 & ~ & ~ \\ \hline
        34A & 0800 & 34B & 0800 & 7FF & F8A5 & 000 & 034A & F8A5 & 1001 & ~ & ~ \\ \hline
        34B & 6E0C & 34C & 6E0C & 358 & F8A4 & 000 & 000C & 0001 & 0001 & ~ & ~ \\ \hline
        34C & EE0B & 34D & EE0B & 358 & 0001 & 000 & 000B & 0001 & 0001 & 358 & 0001 \\ \hline
        34D & AE08 & 34E & AE08 & 356 & 0000 & 000 & 0008 & 0000 & 0101 & ~ & ~ \\ \hline
        34E & 0740 & 34F & 0740 & 34E & 0740 & 000 & 034E & FFFF & 1000 & ~ & ~ \\ \hline
        34F & 0C00 & 350 & 0C00 & 7FF & FFFF & 7FF & 034F & FFFF & 1000 & 7FF & FFFF \\ \hline
        350 & D359 & 359 & D359 & 7FE & 0351 & 7FE & D359 & FFFF & 1000 & 7FE & 0351 \\ \hline
        359 & AC01 & 35A & AC01 & 7FF & FFFF & 7FE & 0001 & FFFF & 1000 & ~ & ~ \\ \hline
        35A & F001 & 35B & F001 & 35A & F001 & 7FE & 035A & FFFF & 1000 & ~ & ~ \\ \hline
        35B & F306 & 35C & F306 & 35B & F306 & 7FE & 035B & FFFF & 1000 & ~ & ~ \\ \hline
        35C & 7E07 & 35D & 7E07 & 364 & 0A00 & 7FE & 0007 & FFFF & 1001 & ~ & ~ \\ \hline
        35D & F804 & 362 & F804 & 35D & F804 & 7FE & 0004 & FFFF & 1001 & ~ & ~ \\ \hline
        362 & AE02 & 363 & AE02 & 365 & F8A5 & 7FE & 0002 & F8A5 & 1001 & ~ & ~ \\ \hline
        363 & EC01 & 364 & EC01 & 7FF & F8A5 & 7FE & 0001 & F8A5 & 1001 & 7FF & F8A5 \\ \hline
        364 & 0A00 & 351 & 0A00 & 7FE & 0351 & 7FF & 0364 & F8A5 & 1001 & ~ & ~ \\ \hline
        351 & 0800 & 352 & 0800 & 7FF & F8A5 & 000 & 0351 & F8A5 & 1001 & ~ & ~ \\ \hline
        352 & 4E05 & 353 & 4E05 & 358 & 0001 & 000 & 0005 & F8A6 & 1000 & ~ & ~ \\ \hline
        353 & EE04 & 354 & EE04 & 358 & F8A6 & 000 & 0004 & F8A6 & 1000 & 358 & F8A6 \\ \hline
        354 & 0100 & 355 & 0100 & 354 & 0100 & 000 & 0354 & F8A6 & 1000 & ~ & ~ \\ \hline
    \end{tabular}
    \caption{Трассировка программы}
\end{table}
