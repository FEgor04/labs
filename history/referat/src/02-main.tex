\chapter{Основы конструктивизма}
\section{Предпосылки возникновения конструктивизма}
Значительные перемены в устройстве общества в начале XX века, а именно Великая Октябрьская Революция,
переход от <<старого капитализма>> к <<новому социализму>>, вызвали изменения и в искусстве.
Наиболее мобильная часть культуры --- художественная среда --- первая начала активно искать
новый язык, позволяющий отражать меняющуюся действительность.

Футуризм, абстракционизм, кубизм, экспрессионизм --- все эти направления демонстрируют отрыв от
прошлого, его рамок и ограничений, декламируя становление нового, свободного общества.

Центральным явлением авангарда явился супрематизм ---
направление в авангардистском искусстве, основанное в 1915 году Казимиром Малевичем.
Супрематизм к 1920-м годам обрел множество сторонников во всех сферах искусства.

Как пишет Золотарева М.В. в своей статье <<Супрематизм, как путь к конструктивизму>>:
<<Основной идеей левых архитекторов, явилась вера
в способность нового искусства и архитектуры изменить
общество, поэтому только новые эстетические категории,
могли стать инструментом этих преобразований.
Все художественные поиски того времени роднит призыв к ``ревизии'' устоявшихся форм,
вычленению их первоэлементов, из которых и сложен окружающий мир, и, наконец, к
их сведению воедино в новом пространстве.
Все более настойчиво выдвигались требования отказа от декорирования, переходу к новой формуле архитектуры, в которой
главными становятся функционализм и рациональность.
Поэтому мировоззрение основоположников супрематизма, работающих с ``первоэлементами мироздания''
становится идейно-художественной платформой архитектуры русского кубофутуризма и в конечном итоге
конструктивизма>>\cite{золотарева2015суприматизм}.

\section{Зарождение конструктивизма}
Как упомянуто выше, конструктивизм берет свое начало в 1920-1930-х годах.
Первое объединение конструктивистов --- \textit{<<Объединение современных архитекторов>>} (ОСА) ---
было основано в 1925 году членами \textit{<<Левого фронта искусства>>}, также известного как ЛЕФ.
Председателем объединения стал Александр Александрович Веснин.

Ввиду существования других объединений архитекторов, таких как
<<Московское архитектурное общество>> и \textit{<<АСНОВА>>} (Ассоциация новых архитекторов)
официально оформить объединение удалось лишь в 1926 году.

Разумеется, первые конструктивисты появились до их формального объединения.
Исходной точкой конструктивизма считается проект
<<Памятника III Интернационала>> --- гигантского здания высотой 400 м.
Проект башни представлял собой совокупность двух наклонных металлических спиралей,
состоящих из расположенных одно над другим зданий различной геометрической формы,
которые кроме всего прочего должны были вращаться вокруг своей оси.
<<Отцом>> конструктивизма считается Владимир Татлин, автор вышеупомянутого проекта~\cite{сидорина2007лики}.



\section{Особенности конструктивизма}
Многие конструктивисты, вдохновляясь примером Каземира Малевича, начали искать выразительность в
самоограничении: попытках выразить свой замысел используя базовые геометрические формы.
Основными особенностями стиля являются строгость, лаконичность форм и монолитность внешнего облика.
\begin{itemize}
    \item \textbf{Монолитность}.
    Архитекторы-конструктивисты стремились к созданию единого образа здания.
    Монументальный Дом Государственной Промышленности, несмотря на внушительную площадь в 10760 $\text{м}^2$ и
    высоту в 63 м кажется единым целым (см. рис.~\ref{fig:gosprom}).

        
    \item \textbf{Сегментированность}.
    Пускай здание и монолитно, но оно разбито на сегменты.
    При этом такое разбиение не приводит к визуальному распаду здания.
    Это отлично видно на уже упомянутом здании Госпрома.
    Улицы разделяют здание на сегменты, но переходы возвращают ему монолитность.


    \item \textbf{Строгость.}
    Стремление к геометризму проявляется в использовании базовых геометрических фигур.
    Почти все конструктивистские работы, от архитектуры до изобразительного искусства,
    составлены из треугольников, прямоугольников и кругов.

    \item \textbf{Рациональность и функциональность.}
    Все конструктивистские работы выполнены в первую очередь для достижения цели.
    Фасад в них становится просто оболочкой, а выразительность переносится на форму здания, на его конструкцию.
    
    
    \item \textbf{Новейшие материалы.}
    Поскольку конструктивисты отвергали <<старый мир>>, они отвергали и его материалы.
    Здания конструктивистов были преимущественно созданы из таких материалов как бетон, металл, стекло.
    Использование новейших материалов также открывало и новейшие формы зданий, совершенно отличные от прежних.
    
\end{itemize}

В данной главе были рассмотрены история конструктивизма, его истоки и особенности.
Данное изучение позволило сделать ряд выводов:

Во-первых, конструктивизм испытал влияние супрематизма, от которого вобрал в себя
строгость и лаконичность форм.

Во-вторых, конструктивизм во многом отражал изменения, происходящие в обществе,
среди которых жажда перемен, переход от старого к новому.

В-третьих, конструктивизм также отражал и новые открытия в науке, например, открытие
железобетона, открытия в области производства стали, стекла и др.
Таким образом, можно сделать вывод о том, что конструктивизм помимо всего прочего
показывает также и то, каким автор представляет будущее и его формы.

% \section{Конструктивизм и социализм}
% Одним из важнейших понятий, введенных Карлом Марксом в своих работах является \textbf{общественное бытие}.
% Маркс считал, что \textit{<<не сознание людей определяет их бытие, а наоборот, их общественное бытие определяет их сознание>>}.
% Как сказано ранее, конструктивисты верили в свою способность изменить общество своими работами.
% Именно поэтому конструктивисты стремились к обобществлению быта.
% 
% Прекрасной демонстрацией этого обобществления является дом-коммуна на улице Орджоникидзе в Москве (см. рис.~\ref{fig:communa}).
% Архитектором здания выступил Иван Сергеевич Николаев.
% Здание представляло из себя по сути <<дом --- машину для жилья>>.
% Оно состояло из трех корпусов: спальный, санитарный и общественный.
% Здание ориентировано на создание жесткого распорядка дня:
% утром студент просыпается в двухместной спальной кабине,
% направляется в санитарный корпус, и затем направляется в общественный корпус, в котором
% размещалась вся необходимая студенту инфраструктура: столовая, спортивный зал, читальный зал,
% детские ясли, прачечные, медпункт, комнаты для кружков и кабины для индивидуальных занятий.
% 
% Наличие такой инфраструктуры позволяло снять с рук студента, а в будущем работника социалистической стройки,
% все <<дела по дому>>, оставив ему только возможность заниматься своим основным делом --- учебой или работой, при этом
% полностью обобществив его занятия.