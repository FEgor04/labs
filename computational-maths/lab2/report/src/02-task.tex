\section{Цель работы}
Цеь работы -- изучить численные методы решения нелинейных уравнений и их
систем, найти корни заданного нелинейного уравнения/системы нелинейных уравнений,
выполнить программную реализацию методов.

% 2 5 3
\section{Вычислительная часть задачи}
\subsection{Решение нелинейного уравнения}
\begin{enumerate}
	\item Отделить корни заданного нелинейного уравнения графически (вид уравнения представлен в табл. 6)
	\item Определить интервалы изоляции корней.
	\item Уточнить корни нелинейного уравнения (см. табл. 6) с точностью $\varepsilon = 10^{-2}$.
	\item Используемые методы для уточнения каждого из 3-х корней многочлена представлены в таблице 7.
	\item Вычисления оформить в виде таблиц (1-5), в зависимости от заданного метода. Для всех значений в таблице удержать 3 знака после запятой.
	\item Для метода хорд заполнить таблицу 2.
	\item Для метода Ньютона заполнить таблицу 3.
	\item Для метода простой итерации заполнить таблицу 5. Проверить условие сходимости метода на выбранном интервале.
	\item Заполненные таблицы отобразить в отчете.
\end{enumerate}

Уравнение:
\[
	5.74 x^3 - 2.95 x^2 - 10.28 x - 3.23
\]

График уравнения доступен по ссылке \url{https://www.desmos.com/calculator/6je61ivycq}.
У уравнения есть три корня.
Начальные интервалы изоляции:
\[
	a_1 = -0.9; b_1 = -0.8
	\qquad
	a_2 = -0.4; b_2 = -0.3
	\qquad
	a_3 = 1.7; b_3 = 1.8
\]

\begin{table}[H]
	% \usepackage{tabularray}
	\begin{longtblr}[
		label = none,
		entry = none,
		]{
		width = \linewidth,
		colspec = {Q[58]Q[58]Q[58]Q[56]Q[112]Q[112]Q[108]Q[113]},
		hlines,
		vlines,
		}
		\textnumero & a     & b   & x     & f(a)   & f(b)  & f(x)   & $|x_{k+1} - x_k|$ \\
		1           & 1.7   & 1.8 & 1.732 & -1.031 & 2.184 & -0.059 & --                \\
		2           & 1.732 & 1.8 & 1.734 & -0.059 & 2.184 & -0.003 & 0.002 < 0.01      \\
	\end{longtblr}
\end{table}

$f' = 17.22 x^2 - 5.9 x - 10.28$.
$f'(-0.9) = 8.9782$.
$f'(-0.8) = 5.4608$.
$\lambda = - \frac{1}{8.9782}$.
$\varphi(x) = x - \frac{1}{8.9782} f(x)$.

\newpage
\begin{table}[H]
	% \usepackage{tabularray}
	\begin{longtblr}[
		label = none,
		entry = none,
		]{
		width = \linewidth,
		colspec = {Q[106]Q[104]Q[100]Q[104]Q[113]},
		hlines,
		vlines,
		}
		\textnumero & $x_k$  & $x_{k+1}$ & $f(x_{k+1})$ & $|x_{k+1} - x_k|$ \\
		1           & -0.9   & -0.839    & -0.018       & 0.061             \\
		2           & -0.839 & -0.831    & -0.018       & -0.008 < 0.01
	\end{longtblr}
\end{table}

Для центрального корня
$a_2 = -0.4, b_2 = -0.3$.

% \usepackage{tabularray}
\begin{longtblr}[
	label = none,
	entry = none,
	]{
	width = \linewidth,
	colspec = {Q[112]Q[108]Q[108]Q[108]Q[108]Q[227]},
	hlines,
	vlines,
	}
	\textnumero & $x_k$ & $f(x_k)$ & $f'(x_k)$ & $x_{k+1}$ & $|x_{k+1}-x_k|$ \\
	1           & -0.4  & 0.043    & -5.165    & -0.392    & 0.008 < 0.01    \\
\end{longtblr}

\subsection{Решение системы уравнений}
\[
	\begin{cases}
		y - \cos x = 2 \\
		x + \cos(y-1) = 0.8
	\end{cases}
\]
Метод простой итерации.
График системы представлен по ссылке \url{https://www.desmos.com/calculator/sutbmcle3l}.
Видно, что ответ лежит в пределах:
\[
	0.5 < x < 1 \qquad 2.6 < y < 2.8
\]

Выразим \(x, y\):
\[
	\begin{cases}
		x = 0.8 - \cos (y-1) \\
		y = 2 + \cos x
	\end{cases}
\]

\[
	\varphi(x, y) = ( 0.8 - \cos y, 2 + \cos x )
\]
\[
	\varphi'(x,y) = \begin{bmatrix}
		0         & \sin(y) \\
		- \sin(x) & 0
	\end{bmatrix}
\]

Значение \(\sin(x)\) на \((0.5, 1)\) меняется в пределах от \(0.479\) до $0.841$,
\(\sin(y)\) на $(2.6, 2.8)$ --- от $0.516$ до $0.335$.
Значит итерационный процесс сходится.

\begin{table}[H]
  \centering
	\begin{tabular}{|c | c|}
		\hline
		$x$      & $y$      \\
		\hline
		$0.5$    & $2.6$    \\
		1.656889 & 2.877583 \\
		1.765351 & 1.914014 \\
		1.136519 & 1.806670 \\
		1.033693 & 2.420755 \\
		1.551253 & 2.511650 \\
		1.608061 & 2.019542 \\
		1.233836 & 1.962744 \\
		1.181989 & 2.330620 \\
		1.488794 & 2.379085 \\
		1.523106 & 2.081911 \\
		1.289149 & 2.047672 \\
		1.259006 & 2.277938 \\
		1.449663 & 2.306763 \\
		1.471304 & 2.120837 \\
		1.322722 & 2.099328 \\
		1.304266 & 2.245538 \\
		1.424696 & 2.263386 \\
		1.438532 & 2.145582 \\
		1.343655 & 2.131879 \\
		1.332103 & 2.225194 \\
		1.408681 & 2.236433 \\
		1.417560 & 2.161406 \\
		1.356868 & 2.152637 \\
		1.349563 & 2.212301 \\
		1.398401 & 2.219433 \\
		1.404101 & 2.171542 \\
		1.365258 & 2.165925 \\
		1.360615 & 2.204094 \\
		1.391806 & 2.208637 \\
		1.395462 & 2.178036 \\
		1.370603 & 2.174437 \\
		1.367644 & 2.198859 \\
		1.387578 & 2.201758 \\
		1.389922 & 2.182195 \\
		1.374013 & 2.179890 \\
		1.372124 & 2.195516 \\
		1.384870 & 2.197368 \\
		1.386371 & 2.184857 \\
		1.376191 & 2.183382 \\
		1.374985 & 2.193379 \\
		\hline
	\end{tabular}
\end{table}

\section{Программная часть задачи}
