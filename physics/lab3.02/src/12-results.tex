\section{Окончательные результаты}
\begin{itemize}
	\item Значения $\varepsilon$ и $r$, полученные с помощью МНК (см.~\cref{fig:u_i}).
	      \[
          \varepsilon = \num{11.6463} \pm \SI{0.0158}{\volt} \: (\SI{0.14}{\percent})
		      \qquad
          r = \num{0.6743} \pm \SI{0.0017}{\ohm} \: (\SI{0.01}{\percent})
	      \]

	\item Графики зависимостей всех мощностей от силы тока представлены на~\cref{fig:p_i}.
	      Значение $I^\star$, при котором полезная мощность достигает максимального значения:
	      \[
		      I^\star = \num{8.88} \, \text{A}
	      \]

	\item Сопротивление $R$, соответствующие режиму согласования нагрузки и источника:
	      \[
		      R = \num{0.6385} \, \text{Ом}
	      \]

	\item График КПД $\eta$ как функции силы тока представлен на рисунке~\cref{fig:eta_i}.
	      Значение $I^*$, соответствующее $\eta = 0.5$:
	      \[
		      I^* = \num{8.6354} \, \text{А}
	      \]
\end{itemize}
