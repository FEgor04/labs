\section{Вывод и анализ результатов работы}
В ходе выполнения данной работы были получены значения логарифмического
декремента при разных значениях сопротивления в магазине.
На графике~\ref{fig:lambda_r} видно, что зависимость логарифмического декремента
от сопротивления является линейной.

Также были получены значение критического сопротивлния \(R\), которое отличается
от теоретического значения на 100 Ом и значение добротности при разном сопротивлении.
Зависимость добротности от сопротивления является экспоненциальной, а 
разность между теоретическми и экспериментальными значениями находится в районе погрешности.

Таким образом, в ходе выполнения лабораторной работы были изучены свободные 
затухающие электромагнитные колебания.
