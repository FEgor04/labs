\section{Расчет погрешности измерений}
Рассчитаем погрешность линейного коэффициента \(a\) зависимости
логарифмическго декремента \(\lambda\) от сопротивления \(R\):
\[
  a^2 = \left( \frac{1}{n} + \frac{(\bar x)^2}{D} \right) \frac{\sum d_i^2}{n-2},
  \qquad 
  D = \sum (x_i - \bar x)^2
  \quad
  d_i = y_i - (a + b x_i)
\]
\[
D = 12 192,
\quad
\sum d_i^2 = 9.017 \cdot 10^{-5}
\quad
a^2 = 3.236 \cdot 10^{-6}
\quad
\Delta a = 2 a = 3.432 \cdot 10^{-3}
\]

Таким образом, погрешность логарифмического декремента:
\[
  \Delta \lambda = 3.432 \cdot 10^{-3}
  \quad
  \varepsilon (\lambda) = 1.02 \%
  \quad
  \alpha = 0.95
\]

Погрешность коэффициента \(b\):
\[
  b^2 = \frac{1/D \cdot \sum d_i^2}{n-2} = 9.1192 \cdot 10^{-10}
  \quad
  \delta b = 2 b = 6.307 \cdot 10^{-5}
\]

Относительная погрешность \(b\):
\[
  \varepsilon(b) = 1.16 \%
  \quad
  \alpha = 0.95
\]

Погрешность \(R\):
\[
  \varepsilon(R) = 2.21 \%
\]

Погрешность \(C\):
\[
  \varepsilon(C) = 10 \%
\]

Погрешность индуктивности \(L\):
\[
  \varepsilon(L) =
  \sqrt{ (2 \varepsilon(R))^2 + (\varepsilon(C))^2 + (-2 \varepsilon(\lambda))^2  } =
  11.12 \%
\]

  
