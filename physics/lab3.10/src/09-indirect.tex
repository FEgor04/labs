\section{Расчет результатов косвенных измерений}


График зависимости логарифического декремента \(\lambda\) от
сопротивления магазина \(R_\text{м}\) представлен на рисунке~\ref{fig:lambda_r}.

Собственное сопротивление контура:
\[ R_0 = - R_{M \mid \lambda = 0} \approx 61.8388 \, \text{Ом} \]

Полное сопротивление контура \(R\) и индуктивность \(L\) при \(R_\text{м}\) = 50 Ом:
\[
  R = R_{\text{м}} + R_0 = 50 + 61.8338 = 111.8338
\]
\[
  L = \frac{\pi^2 R^2 C}{\lambda^2} \approx 1.44
\]

Среднее значение индуктивности: \(L_\text{ср} = 7.3745\) Гн.

Сравнение периода колебаний, полученных экспериментальным и теоретическим
способом представлены на таблице~\ref{table:t_theor_exp}.

\begin{table}[H]
% \usepackage{tabularray}
\centering
\begin{longtblr}[
  label = none,
  entry = none,
]{
  width = \linewidth,
  colspec = {Q[100]Q[100]Q[100]},
  hlines,
  vlines
}
R, Ом & \(T_\text{теор}\), мс & \(T_\text{эксп}\), мс \\
 0 &    0.09063 &  0.09 \\
 200 &  0.09249 & 0.09 \\
 400 &  0.09683 & 0.09 
\end{longtblr}
\caption{Сравнение теор. и эксп. периода колебаний в контуре при разных сопротивлениях}
\label{table:t_theor_exp}
\end{table}


График зависимости добротности контура \(Q\) от сопротивления магазина
представлен на рисунке~\ref{fig:q_r}.
Добротность контура по формуле \( Q = \frac{1}{R} \cdot \sqrt{\frac{L}{C}} \) 
для \(R = 0\) Ом \(Q = 9.27\), для \(R = 50\) Ом \(Q = 5.12\).

Экспериментальное критическое сопротивление контура \(R \approx 1211.8338 \) Ом,
теоретическое сопротивление, полученное по формуле
\( R_\text{кр} = 2 \cdot \sqrt{\frac{L}{C}} \approx 1311.3392 \).

Графики зависимости периодов \(T_\text{эксп}\) и \(T_\text{теор}\) представлены
на рисунке~\ref{fig:t_c}.
Видно, что графики не имеют сильных расхождений в значениях, а
значит в данном случае можно использовать формулу Томсона и
выполняется условие \(\beta \ll \omega\).
