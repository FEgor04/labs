\clearpage
\section{Расчет результатов косвенных измерений}
График зависимости амплитуды колебаний $A(t)$ от времени представлен
на рисунке \ref{fig:ampl_t}. 
В затухании главную роль играет вязкое трение.
График, соответствующий формуле $ \ln (A / A_0) = - \beta t $ 
представлен на рисунке~\ref{fig:ln}.
Коэффициент затухания $\beta \approx 0.00638$,
время затухания $\tau = 1/\beta \approx 156.77$ с.

График $T^2(I)$ представлен на рис.~\ref{fig:t_squared}.
По методу наименьших квадратов была проведена аппроксимирующая прямая.
Ее угловой коэффициент $k \approx 86.70087$.
Таким образом, $ml = (4 \pi^2)/(kg) \approx 0.04637$.


\begin{table}[ht]
% \usepackage{tabularray}
\begin{longtblr}[
  label = none,
  entry = none,
]{
  cell{2}{2} = {c=6}{},
  cell{3}{2} = {c=6}{},
  hlines,
  vlines,
}
Риски                      & 1      & 2      & 3     & 4     & 5     & 6     \\
$R_\text{в}$, м            & 0.077  &        &       &       &       &       \\
$R_\text{н}$, м            & 0.202  &        &       &       &       &       \\
$R_\text{б}$, м            & 0.077  & 0.102  & 0.127 & 0.152 & 0.177 & 0.202 \\
$I_\text{гр}$, Н $\cdot$ м & 0.0239 & 0.0276 & 0.0322 & 0.0379 & 0.0446 & 0.0524 \\
$I$, Н $\cdot$ м           & 0.0319 & 0.0356 & 0.0402 & 0.0459 & 0.0526 & 0.0604\\
$l_\text{пр}$ эксп, м      & 0.6796 & 0.7557 & 0.8577 & 0.9860 & 1.1316 & 1.2885 \\
$l_\text{пр}$ теор, м      & 0.6880 & 0.7668 & 0.8675 & 0.9903 & 1.1350 & 1.3017
\end{longtblr}
\caption{Результаты вычисления $I$ и $l$ в зависимости от положения грузов}
\end{table}
