\section{Окончательные результаты}
\begin{itemize}
  \item Период и циклическая частота свободных колебаний маятника:
  \[
  T = 1.7657 \pm 0.01973 \, \text{с}
  \qquad
  \omega_0 = 3.5585 \pm 0.0397 \, \frac{\text{рад}}{\text{с}}
  \]
  \item Графики $A(t)$ зависимости амплитуды колебаний от времени
  для разных токов электромагнитного тормоза представлены на рис.~\ref{fig:amplitude_t}.

  \item Графики $f(t) = \ln(A_0 / A_k)$ представлены на рис.~\ref{fig:log_dec}. 
  Значения параметров $\lambda, \beta, Q$ представлены в табл.~\ref{table:lambda_beta_q}.
  
  \item Градуировочный график $\omega(U)$ представлен на рис.~\ref{fig:omega_u}.
  
  \item Графики АЧХ для трех коэффициентов затухания представлены на рисунке~\ref{fig:a_omega}.

  \item Сравнение теоретической и экспериментальной АЧХ для $I_T = 200$ мА представлены на 
  рис.~\ref{fig:theor_vs_emp}.
  Видно, что на области снятых измерений графики сильно напоминают друг друга.

  \item Сравнение значений добротности, найденных по затуханию колебаний и по АЧХ
  представлены в табл.~\ref{table:a_omega}.
  \end{itemize}
