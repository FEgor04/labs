\section{Расчет результатов косвенных измерений}
График зависимости напряжения выходного сигнала от частоты входного
представлен на рисунке~\ref{fig:u_f}.
На графике видно, что резонансная частота равна 4750 Гц.
Расчетная частота при этом равна 5000 Гц.

Из-за смещения $f_{res}$ теоретической и практической, на график не полностью
попал график на уровне $\frac{1}{\sqrt{2}} \Omega_0$.
Поэтому в дальнейшем $\Delta \Omega$ взята за $2 \cdot (5500 - 4750) = 1500$ Гц.
Тогда $Q = \frac{\Omega_0}{\Delta \Omega} = {4750}{1500} \approx 3.16$.

Оценим добротность контура с помощью ЭДС.
\[
  R = \frac{U}{\varepsilon} = \frac{9.2}{0.3} = 30 \, \text{Ом}
\]
\[
Q' = \frac{1}{R} \sqrt{\frac{L}{C}} = \frac{1}{30} \cdot 100 = 3.33
\]

Значение добротности, полученное по графику, отличается от значения добротности,
полученного с помощью ЭДС на \(3.33 - 3.16 = 0.17\) или на 5 \% от \(Q'\).

График зависимости квадрата резонансной частоты представлен на рисукне~\ref{fig:res_c}.
С помощью метода наименьшего квадрата были получены следующие значения:
\[
  \frac{1}{L} = 248.90 \Rightarrow L = 0.004 \, \text{Гн}
\]
Значение активного сопротивления:
\[
  R = \sqrt{- 4 L^2 b} = 658 \, \text{Ом}
\]
