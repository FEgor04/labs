\documentclass{article}
\usepackage[T2A]{fontenc}
\usepackage{amsmath,amsthm} 
\usepackage{float} 
\usepackage[utf8]{inputenc}
\usepackage{hyperref}
\usepackage[english,russian]{babel}
\everymath{\displaystyle}
\usepackage[usenames]{color}
\usepackage{graphicx}
\usepackage{enumitem}
\usepackage{epigraph}
\usepackage{xcolor}

\usepackage{geometry}
\geometry{
    a4paper,
    top=20mm, 
    right=1cm,
    bottom=20mm, 
    left=1cm
}

\usepackage{fancyhdr}
\usepackage{amsfonts}
\pagestyle{fancy}
\fancyhead{}
\fancyfoot[C]{\thepage}

\renewcommand{\headrulewidth}{0pt}
\usepackage{tikz}
\usepackage{tikzsymbols}
\usepackage{textcomp,latexsym,pb-diagram,amsopn}
\usepackage{gnuplottex}

\newtheoremstyle{problemstyle}  % <name>
{3pt}                                               % <space above>
{3pt}                                               % <space below>
{\normalfont}                               % <body font>
{}                                                  % <indent amount}
{\bfseries\itshape}                 % <theorem head font>
{\normalfont\bfseries:}         % <punctuation after theorem head>
{.5em}                                          % <space after theorem head>
{}                                                  % <theorem head spec (can be left empty, meaning `normal')>
\theoremstyle{problemstyle}
\newtheorem{problem}{Задача}[section]

\newcommand\lword[1]{\leavevmode\nobreak\hskip0pt plus\linewidth\penalty50\hskip0pt plus-\linewidth\nobreak\textbf{#1}}
\title{\textbf{Теория вероятностей} \\ Домашнее задание \textnumero 2}
\author{Федоров Егор, P3215, вариант 19}
\date{}

\begin{document}
\maketitle
\section{ИДЗ 19.1}
В результате эксперимента получены данные, записанные в виде статистического ряда.
Требуется:
\begin{enumerate}
	\item Записать значения результатов эксперимента в виде вариационного ряда;
	\item Найти размах варьирования и разбить его на 9 интервалов;
	\item Построить полигон частот, гистограмму относительных частот и график
	      эмпиричесокй функции распределения;
	\item Найти числовые характеристики выборки \(\bar x\) и \(D_\text{в}\);
	\item Приняв в качестве нулевой гипотезу \(H_0\): генеральная совокупность,
	      из которой извлечена выборка, имеет нормальное распределение, проверить ее,
	      пользуясь критерием Пирсона при уровне значимости \(\alpha = 0.025\);
	\item Найти доверительные интервалы для математического ожидания и среднего
	      квадратичного отклонения при надежности \(\gamma = 0.9\);
\end{enumerate}

\subsection{Решение}
\begin{enumerate}
	\item Вариационный ряд для данный выборки представлен на табл.~\ref{table:var_series}.
	\item Размах варьирования \(\omega = x_{\max} - x_{\min} = 80.8 - 10.8 = 70\).
	      Длина частотного интервала \(h = \omega / l\), где \(l = 9\).
	      Тогда \(h = 70/9 \approx 7.78\).
	      Границы интервалов и их середины представлены на табл.~\ref{table:intervals}.
	\item Полигон частот, гистограмма относительных частот и
	      график эмпирической функции распределения представлены на рисунках
	      \ref{fig:polygon}, \ref{fig:hist}, \ref{fig:emp_func}.
	\item
	      \[ \bar x = \sum_{i = 1}^n \frac{x_i}{n} = 46.243 \]
	      \[
		      D_\text{в} =
		      \frac{1}{n} \sum_{i=1}^n \left( x_i^2 - \bar x^2 \right) =
		      364.7415
		      \qquad
		      \sigma_\text{в} =
		      \sqrt{D_\text{в}} \approx
		      19.0982
	      \]
	      \[ S^2 = \frac{n}{n-1} D_\text{в} = 368.4257 \]
	\item Перейдем к случайной величине \(z = (x - \bar x) / \sigma_\text{в} \).
	      Вычислим концы интервалов \(z_i = (x_i - \bar x) \ sigma_\text{в} \),
	      причем $z_1$ положим стремящимся к $-\infty$, а $z_{m+1} \to +\infty$.
	      Вычисления теоретических вероятностей представлены на таблице~\ref{table:theoretical} и \ref{table:theoretical_2}.
	      Наблюдаемое значение критерия Пирсона:
	      \begin{align*}
		      \chi_\text{набл}^2 & =
		      \sum_{i=1}^9 \frac{(n_i - n^\prime_i)^2}{ n_i^{\prime2}} \approx
		      \frac{(6 - 6.81)^2}{46.38} + \frac{(13 - 8.1)^2}{65.61} + \frac{(11 - 11.5)^2}{132.2} + \frac{(11 - 14.87)^2}{221.1} + \frac{(15 - 15.83)^2}{250.6} +                                                                          \\ &+ \frac{(14 - 14.76)^2}{217.9} + \frac{(8 - 11.98)^2}{143.5} + \frac{(13 - 8.03)^2}{64.48} + \frac{(9 - 8.076)^2}{65.22} = \\
		                         & = \frac{0.6561}{46.38} + \frac{24.01}{65.61} + \frac{0.25}{132.2} + \frac{14.98}{221.1} + \frac{0.6889}{250.6} + \frac{0.5776}{217.9} + \frac{15.84}{143.5} + \frac{24.7}{64.48} + \frac{0.8538}{65.22} = \\
		                         & = 0.01415 + 0.366 + 0.00189 + 0.06773 + 0.002749 + 0.002651 + 0.1104 + 0.3831 + 0.01309                                                                                                                   \\
		                         & = 0.9617
	      \end{align*}
	      Уровню значимости \(\alpha = 0.025\) и числу степеней свободы \(k = l - 3 = 6\)
	      соответствует $\chi^2_\text{крит} = 14.4 $.

	      Таким образом, \(\chi^2_\text{набл} < \chi^2_\text{крит}\), а значит гипотеза
	      \(H_0\) принимается.

	\item Так как \(X\) распределена нормально, то с надежностью \(\gamma\)
	      можно утверждать, что математическое ожидание \(a\) покрывается
	      доверительным интервалом
	      \[
		      \left(
		      \bar x - \frac{\tilde\sigma_\text{в}}{\sqrt{n}} t_\gamma
		      ;
		      \bar x + \frac{\tilde\sigma_\text{в}}{\sqrt{n}} t_\gamma
		      \right)
	      \]
	      \[
		      t_\gamma = 1.65
		      \qquad
		      \tilde\sigma_\text{в} = \sqrt{S^2} = \sqrt{368.4257}
		      \qquad
		      n = 10
	      \]
	      \[
		      \frac{\tilde\sigma_\text{в}}{\sqrt{n}} t_\gamma = \frac{\sqrt{368.4257}}{10} 1.65 \approx 3.167
		      \qquad
		      \bar x = 46.243
	      \]
	      Таким образом, искомый доверительный интервал:
	      \[
		      (43.075; 49.41)
	      \]

	      Для оценки $\sigma$ с уверенностью $\gamma$ используем
	      доверительный интервал \( (\tilde\sigma_\text{в} (1 - q); \tilde\sigma_\text{в} (1+q)) \)
	      При $n = 100$ и $\gamma = 0.9$, $q = 0.143$. Тогда искомый доверительный интервал:
	      \[ (16.4496; 21.9392) \]
\end{enumerate}

\begin{table}[H]
	\centering
	\begin{tabular}{|l|l|l|l|l|l|l|l|l|l|}
		\hline
		19.3 & 44.5 & 49.9 & 26.9 & 50.2          & 51.1 & 18.6 & 72.7 & 35.4 & 25.4 \\
		\hline
		42.7 & 17.5 & 51.7 & 49.3 & 26.2          & 47.1 & 71.4 & 27.1 & 75.7 & 43.2 \\
		\hline
		25.5 & 27.2 & 80.4 & 50.4 & 70.2          & 14.9 & 52.4 & 62.3 & 41.7 & 49.5 \\
		\hline
		40.6 & 14.5 & 62.8 & 34.5 & 53.4          & 26.1 & 69.3 & 52.5 & 27.3 & 80.3 \\
		\hline
		25.3 & 43.1 & 27.4 & 80.1 & 68.4          & 63.3 & 13.4 & 55.4 & 39.5 & 33.1 \\
		\hline
		38.4 & 19.7 & 63.8 & 40.4 & \textbf{80.8} & 56.4 & 66.1 & 27.5 & 79.1 & 24.6 \\
		\hline
		28.6 & 47.9 & 78.4 & 57.4 & 66.5          & 37.3 & 23.4 & 67.6 & 11.1 & 64.3 \\
		\hline
		22.7 & 64.8 & 36.2 & 58.7 & \textbf{10.8} & 47.7 & 58.4 & 29.2 & 46.7 & 77.2 \\
		\hline
		51.9 & 31.3 & 44.7 & 66.3 & 20.1          & 65.3 & 45.5 & 76.3 & 67.8 & 35.1 \\
		\hline
		66.9 & 18.9 & 42.9 & 50.7 & 34.9          & 43.5 & 32.5 & 48.4 & 53.1 & 65.8 \\
		\hline
	\end{tabular}
	\caption{Данные, полученные в результате эксперимента}\label{table:data}
\end{table}

\begin{table}[H]
	\centering
	\begin{tabular}{|l|l|l|l|l|l|l|l|l|l|}
		\hline
		10.8 & 11.1 & 13.4 & 14.5 & 14.9 & 17.5 & 18.6 & 18.9 & 19.3 & 19.7 \\
		\hline
		20.1 & 22.7 & 23.4 & 24.6 & 25.3 & 25.4 & 25.5 & 26.1 & 26.2 & 26.9 \\
		\hline
		27.1 & 27.2 & 27.3 & 27.4 & 27.5 & 28.6 & 29.2 & 31.3 & 32.5 & 33.1 \\
		\hline
		34.5 & 34.9 & 35.1 & 35.4 & 36.2 & 37.3 & 38.4 & 39.5 & 40.4 & 40.6 \\
		\hline
		41.7 & 42.7 & 42.9 & 43.1 & 43.2 & 43.5 & 44.5 & 44.7 & 45.5 & 46.7 \\
		\hline
		47.1 & 47.7 & 47.9 & 48.4 & 49.3 & 49.5 & 49.9 & 50.2 & 50.4 & 50.7 \\
		\hline
		51.1 & 51.7 & 51.9 & 52.4 & 52.5 & 53.1 & 53.4 & 55.4 & 56.4 & 57.4 \\
		\hline
		58.4 & 58.7 & 62.3 & 62.8 & 63.3 & 63.8 & 64.3 & 64.8 & 65.3 & 65.8 \\
		\hline
		66.1 & 66.3 & 66.5 & 66.9 & 67.6 & 67.8 & 68.4 & 69.3 & 70.2 & 71.4 \\
		\hline
		72.7 & 75.7 & 76.3 & 77.2 & 78.4 & 79.1 & 80.1 & 80.3 & 80.4 & 80.8 \\
		\hline
	\end{tabular}
	\caption{Вариационный ряд}\label{table:var_series}
\end{table}

\begin{table}[H]
	\centering
	\begin{tabular}{|l|l|l|l|l|l|l|}
		\hline
		\(i\) & \(x_i\) & \(x_{i+1}\) & \(x'_i = (x_i + x_{i+1})/2\) & \(n_i\) & \(W_i = n_i/n\) & \(W_i/h\) \\
		\hline
		1     & 10.8000 & 18.5778     & 14.6889                      & 6       & 0.0600          & 0.0077    \\
		\hline
		2     & 18.5778 & 26.3556     & 22.4667                      & 13      & 0.1300          & 0.0167    \\
		\hline
		3     & 26.3556 & 34.1333     & 30.2444                      & 11      & 0.1100          & 0.0141    \\
		\hline
		4     & 34.1333 & 41.9111     & 38.0222                      & 11      & 0.1100          & 0.0141    \\
		\hline
		5     & 41.9111 & 49.6889     & 45.8000                      & 15      & 0.1500          & 0.0193    \\
		\hline
		6     & 49.6889 & 57.4667     & 53.5778                      & 14      & 0.1400          & 0.0180    \\
		\hline
		7     & 57.4667 & 65.2444     & 61.3556                      & 8       & 0.0800          & 0.0103    \\
		\hline
		8     & 65.2444 & 73.0222     & 69.1333                      & 13      & 0.1300          & 0.0167    \\
		\hline
		9     & 73.0222 & 80.8000     & 76.9111                      & 9       & 0.0900          & 0.0116    \\
		\hline
	\end{tabular}
	\caption{Распределение данных по интервалам}\label{table:intervals}
\end{table}

\begin{table}[H]
	\centering
	\begin{tabular}{|l|l|l|l|l|l|l|}
		\hline
		\(i\) & \(x_i\)                        & \(x_{i+1}\)                             & \(x_i - \bar x\) & \(x_{i+1} - \bar x\)
		      & \(z_i = (x_i -\bar x)/\sigma\) & \(z_{i+1} = (x_{i+1} - \bar x)/\sigma\)                                                               \\
		\hline
		1     & 10.8000                        & 18.5778                                 & ---              & -27.6652             & ---     & -1.4486 \\
		\hline
		2     & 18.5778                        & 26.3556                                 & -27.6652         & -19.8874             & -1.4486 & -1.0413 \\
		\hline
		3     & 26.3556                        & 34.1333                                 & -19.8874         & -12.1097             & -1.0413 & -0.6341 \\
		\hline
		4     & 34.1333                        & 41.9111                                 & -12.1097         & -4.3319              & -0.6341 & -0.2268 \\
		\hline
		5     & 41.9111                        & 49.6889                                 & -4.3319          & 3.4459               & -0.2268 & 0.1804  \\
		\hline
		6     & 49.6889                        & 57.4667                                 & 3.4459           & 11.2237              & 0.1804  & 0.5877  \\
		\hline
		7     & 57.4667                        & 65.2444                                 & 11.2237          & 19.0014              & 0.5877  & 0.9949  \\
		\hline
		8     & 65.2444                        & 73.0222                                 & 19.0014          & 26.7792              & 0.9949  & 1.4022  \\
		\hline
		9     & 73.0222                        & 80.8000                                 & 26.7792          & ---                  & 1.4022  & ---     \\
		\hline
	\end{tabular}
	\caption{Вычисление теоретических частот}\label{table:theoretical}
\end{table}

\begin{table}[H]
	\centering
	\begin{tabular}{|l||l|l||l|l||l|l|}
		\hline
		\(i\)                               & \(z_i\)           & \(z_{i+1}\) &
		\(\Phi(z_i)\)                       & \(\Phi(z_{i+1})\) &
		\(P_i = \Phi(z_{i+1}) - \Phi(z_i)\) &
		\(n'_i = P_i \cdot n\)
		\\
		\hline
		1                                   & ---               & -1.4486     & -0.5    & -0.4319 & 0.0681  & 6.81  \\
		\hline
		2                                   & -1.4486           & -1.0413     & -0.4319 & -0.3508 & 0.081   & 8.1   \\
		\hline
		3                                   & -1.0413           & -0.6341     & -0.3508 & -0.2357 & 0.1151  & 11.51 \\
		\hline
		4                                   & -0.6341           & -0.2268     & -0.2357 & -0.0870 & 0.1487  & 14.87 \\
		\hline
		5                                   & -0.2268           & 0.1804      & -0.0870 & 0.0714  & 0.1583  & 15.83 \\
		\hline
		6                                   & 0.1804            & 0.5877      & 0.0714  & 0,21904 & 0.1476  & 14.76 \\
		\hline
		7                                   & 0.5877            & 0.9949      & 0.21904 & 0,33891 & 0.1198  & 11.98 \\
		\hline
		8                                   & 0.9949            & 1.4022      & 0,33891 & 0,41924 & 0.0803  & 8.03  \\
		\hline
		9                                   & 1.4022            & ---         & 0,41924 & 0.5     & 0.08076 & 8.076 \\
		\hline
	\end{tabular}
	\caption{Вычисление теоретических частот}\label{table:theoretical_2}
\end{table}





\begin{figure}[H]
	\centering
	\includegraphics[width=0.6\textwidth]{polygon.pdf}
	\caption{Полигон частот}\label{fig:polygon}
\end{figure}

\begin{figure}[H]
	\centering
	\includegraphics[width=0.6\textwidth]{hist.pdf}
	\caption{Гистограмма относительных частот}\label{fig:hist}
\end{figure}

\begin{figure}[H]
	\centering
	\includegraphics[width=0.6\textwidth]{emp_func.pdf}
	\caption{График эмпирической функции распределения}\label{fig:emp_func}
\end{figure}

\clearpage

\section{ИДЗ 19.2}
Дана таблица распределения 100 заводов по производственным средствам
$X$ (тыс. ден. ед.) и по суточной выработке $Y$ (т.).
Известно, что между $X$ и $Y$ существует линейная корреляционная зависимость.
Требуется:
\begin{itemize}
	\item Найти уравнение прямой регрессии $y$ на $x$.
	\item Построить уравнение эмпирической линии регрессии и случайные точки выборки
	      $(X, Y)$.
\end{itemize}

\subsection{Решение}
Найдем выборочные средние $\bar x$ и $\bar y$.
\begin{align*}
	\bar x & =
	\frac{1}{n} \sum_{i=1}^6 {m_x}_i x_i =
	\frac{1}{100} \left( 11 \cdot 120 + 13 \cdot 130 + 15 \cdot 140 + 26 \cdot 150 + 160 \cdot 20 + 15 \cdot 170 \right) =
	\frac{14760}{100} =
	147.6
\end{align*}
\[
	\bar y =
	\frac{1}{n} \sum_{j=1}^8 {m_y}_j y_j =
	\frac{1}{100} \left( 5 \cdot 8.0 + 9 \cdot 8.8 + 9.6 \cdot 8 + 17 \cdot 10.4 + 19 \cdot 11.2 + 13 \cdot 12.0 + 16 \cdot 12.8 + 13 \cdot 13.6 \right) =
	\frac{1123.2}{100} = 11.232
\]
Выборочная дисперсия $s_x^2$:
\begin{align*}
	s_x^2 & =
	\frac{1}{n-1} \left(\sum {m_x}_i x_i^2 - \frac{1}{100} \left(\sum {m_x}_i{x_i}\right)^2\right) =
	\frac{1}{n-1} \left(\sum {m_x}_i x_i^2 - \frac{1}{100} \left(100 \bar x\right)^2\right) =                                                                                                              \\
	      & = \frac{1}{99} \left[ \left( 11 \cdot 120^2 + 13 \cdot 130^2 + 15 \cdot 140^2 + 26 \cdot 150^2 + 20 \cdot 160^2 + 15 \cdot 170^2 \right) - \frac{1}{100} \left( 100 \bar x \right)^2 \right] = \\
	      & = \frac{1}{99} \left[ 2202600 - \frac{1}{100} 14760^2 \right] =                                                                                                                                \\
	      & = \frac{728}{3} = 242.66
\end{align*}
% \[
%   s_y^2 \approx \frac{11751}{99} \approx 118.7
% \]

Для вычисления корреляционного момента сначала вычислим сумму $ \sum \sum m_{ij} x_i y_j $:
\begin{align*}
	\sum \sum m_{ij} x_i y_j & = 5 \cdot 120 \cdot 8.0 + 6 \cdot 120 \cdot 8.8 +
	3 \cdot 130 \cdot 8.8 + 4 \cdot 130 \cdot 9.6 + 6 \cdot 130 \cdot 10.4    +                              \\
	                         & + 4 \cdot 140 \cdot 9.6 + 5 \cdot 140 \cdot 10.4 + 6 \cdot 140 \cdot 11.2 +
	6 \cdot 150 \cdot 10.4 + 13 \cdot 150 \cdot 11.2 + 7 \cdot 150 \cdot 12.0 +                              \\
	                         & + 6 \cdot 160 \cdot 12.0 + 9 \cdot 160 \cdot 12.8 + 5 \cdot 160 \cdot 13.6  +
	7 \cdot 170 \cdot 12.8 + 8 \cdot 170 \cdot 13.6 =                                                        \\
	                         & = 168096
\end{align*}

Корреляционный момент $s_{xy}$:


\begin{align*}
	s_{xy} & =
	\frac{1}{n-1} \left( \sum \sum m_{ij} x_i y_j - \frac{1}{n} \left(\sum {m_x}_i x_i \right)\left(\sum {m_y}_j y_j \right) \right) = \\
	       & = \frac{1}{n-1} \left( \sum \sum m_{ij} x_i y_j - \frac{1}{n} (100 \bar x) (100 \bar y) \right) =                         \\
	       & = \frac{1}{99} \left( 168096 - \frac{1}{100} (14760) (1123.2) \right) =                                                   \\
	       & = 23.3503
\end{align*}

Тогда зависимость $y$ от $x$ выражается формулой:
\[
	y =
	\bar y + r_{xy} \frac{s_y}{s_x} (x - \bar x) =
	\bar y + \frac{s_{xy}}{s_x s_y} \frac{s_y}{s_x} (x - \bar x) =
	\bar y + \frac{s_{xy}}{s_x^2} (x - \bar x) =
\]

Подставляя полученные ранее значения, получаем:
\[
	y = 11.232 + \frac{23.3503}{242.66} (x - 147.6) =
	0.096 x - 2.971
\]
График линии регрессии представлен на рисунке~\ref{fig:regression}

\begin{table}[H]
	\centering
	\begin{tabular}{|l|l|l|l|l|l|l|l|l|l|}
		\hline
		$X \setminus Y$ & 8.0 & 8.8 & 9.6 & 10.4 & 11.2 & 12.0 & 12.8 & 13.6 & $m_x$ \\
		\hline
		120             & 5   & 6   & -   & -    & -    & -    & -    & -    & 11    \\
		130             & -   & 3   & 4   & 6    & -    & -    & -    & -    & 13    \\
		140             & -   & -   & 4   & 5    & 6    & -    & -    & -    & 15    \\
		150             & -   & -   & -   & 6    & 13   & 7    & -    & -    & 26    \\
		160             & -   & -   & -   & -    & -    & 6    & 9    & 5    & 20    \\
		170             & -   & -   & -   & -    & -    & -    & 7    & 8    & 15    \\
		\hline
		$m_y$           & 5   & 9   & 8   & 17   & 19   & 13   & 16   & 13   & 100   \\
		\hline
	\end{tabular}
	\caption{Таблица распределения}\label{tab:2_data}
\end{table}

\begin{figure}[H]
	\centering
	\includegraphics[width=0.9\textwidth]{regression.pdf}
	\caption{Линия регрессии и случайные точки}\label{fig:regression}
\end{figure}


\end{document}
