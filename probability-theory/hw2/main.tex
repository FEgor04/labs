\documentclass{article}
\usepackage[T2A]{fontenc}
\usepackage{amsmath,amsthm} 
\usepackage[utf8]{inputenc}
\usepackage{hyperref}
\usepackage[english,russian]{babel}
\everymath{\displaystyle}
\usepackage[usenames]{color}
\usepackage{graphicx}
\usepackage{enumitem}
\usepackage{epigraph}
\usepackage{xcolor}

\usepackage{geometry}
\geometry{
    a4paper,
    top=20mm, 
    right=1cm,
    bottom=20mm, 
    left=1cm
}

\usepackage{fancyhdr}
\usepackage{amsfonts}
\pagestyle{fancy}
\fancyhead{}
\fancyfoot[C]{\thepage}

\renewcommand{\headrulewidth}{0pt}
\usepackage{tikz}
\usepackage{tikzsymbols}
\usepackage{textcomp,latexsym,pb-diagram,amsopn}
\usepackage{gnuplottex}

\newtheoremstyle{problemstyle}  % <name>
{3pt}                                               % <space above>
{3pt}                                               % <space below>
{\normalfont}                               % <body font>
{}                                                  % <indent amount}
{\bfseries\itshape}                 % <theorem head font>
{\normalfont\bfseries:}         % <punctuation after theorem head>
{.5em}                                          % <space after theorem head>
{}                                                  % <theorem head spec (can be left empty, meaning `normal')>
\theoremstyle{problemstyle}
\newtheorem{problem}{Задача}[section]

\newcommand\lword[1]{\leavevmode\nobreak\hskip0pt plus\linewidth\penalty50\hskip0pt plus-\linewidth\nobreak\textbf{#1}}
\title{\textbf{Теория вероятностей} \\ Домашнее задание \textnumero 2}
\author{Федоров Егор, P3215, вариант 19}
\date{}

\begin{document}
\maketitle
\section{ИДЗ-19.1}
В результате эксперимента получены данные, записанные в виде статистического ряда.
Требуется:
\begin{enumerate}
  \item Записать значения результатов эксперимента в виде вариационного ряда;
  \item Найти размах варьирования и разбить его на 9 интервалов;
  \item Построить полигон частот, гистограмму относительных частот и график 
    эмпиричесокй функции распределения;
  \item Найти числовые характеристики выборки \(\bar x\) и \(D_\text{в}\);
  \item Приняв в качестве нулевой гипотезу \(H_0\): генеральная совокупность,
    из которой извлечена выборка, имеет нормальное распределение, проверить ее,
    пользуясь критерием Пирсона при уровне значимости \(\alpha = 0.25\);
  \item Найти доверительные интервалы для математического ожидания и среднего
    квадратичного отклонения при надежности \(\gamma = 0.9\);

\end{enumerate}
\end{document}
