\documentclass{article}
\usepackage[T2A]{fontenc}
\usepackage{amsmath} 
\usepackage[utf8]{inputenc}
\usepackage{hyperref}
\usepackage[english,russian]{babel}
\everymath{\displaystyle}
\usepackage[usenames]{color}
\usepackage{graphicx}
\usepackage{minted}

\usepackage{epigraph}

\usepackage{geometry}
\geometry{
  a4paper,
    top=25mm, 
    right=2cm,
    bottom=25mm, 
    left=2cm
}

\newcommand{\lin}{Lin}

\usepackage{fancyhdr}
\usepackage{amsfonts}
\pagestyle{fancy}
\fancyhead{}
\fancyfoot[C]{\thepage}

\renewcommand{\headrulewidth}{0pt}
\usepackage{tikzsymbols}

\newcommand\lword[1]{\leavevmode\nobreak\hskip0pt plus\linewidth\penalty50\hskip0pt plus-\linewidth\nobreak\textbf{#1}}
\title{Отчет о выполнении практической работы \textnumero 5}
\author{Федоров Егор, P3215, вариант 19}
\date{}

\begin{document}
  \maketitle

  Необходимо определить следующие статистические
  характеристики:
  \begin{itemize}
    \item вариационный ряд
    \item экстремальные значения и размах
    \item оценки математического ожидания и среднеквадратического отклонения
    \item эмпирическую функцию распределения и её график
    \item гистограмму и полигон приведенных частот группированной выборки.
  \end{itemize}

  Для расчета
  характеристик и построения графиков нужно написать программу на одном из языков
  программирования. Листинг программы и результаты работы должны быть представлены в отчете
  по практической работе.

  При выполнении практической работы были получены следующие результаты:
  \begin{itemize}
    \item Вариационный ряд:
    \[
      -1.55 \leq -1.45 \leq -1.45 \leq -1.38 \leq -1.31 \leq -1.14 \leq -1.0
      \leq -0.9 \leq 0.17 \leq 0.24   \leq
    \]
    \[
      \leq 0.34 \leq 0.38 \leq 0.52 \leq 0.52 \leq 0.55 \leq 0.62 \leq 0.73
      \leq 0.8 \leq 0.9 \leq 1.31
    \]
    \item Экстремальные значения:
    \[
      x_{(20)} = 1.31 \qquad x_{(1)} = -1.55 \qquad R = x_{(n)} - x_{(1)} = 2.86
    \]
    \item Оценки математического ожидания и среднеквадратичного отклонения:
    \[ MX \approx \bar x = \frac{1}{n} \sum_{i=1}^n x_i = -0.1550 \]
    \[ DX \approx \frac{1}{n} \sum_{i=1}^n \left(x_i - \bar x \right)^2 = 0.9065 \]
    \[ \sigma = \sqrt{DX} \approx 0.9521  \]
    \[ S^2 \approx 0.9542 \]
    \[ S \approx 0.9768 \]
    \item Эмпирическая функция распределния описана в таблице \ref{table:emp}.
    График эмпирической функции распределния представлен на рисунке \ref{fig:emp}.
    Эмпирическая функция в аналитическом виде:
      \[
        F^*(x) = \begin{cases}
          0,      & x \leq -1.55 \\
          0.05, & -1.55 < x \leq -1.45 \\
          0.15, & -1.45 < x \leq -1.38 \\
          0.2, & -1.38 < x \leq -1.31 \\
          0.25, & -1.31 < x \leq -1.14 \\
          0.3, & -1.14 < x \leq -1.0 \\
          0.35, & -1.0 < x \leq -0.9 \\
          0.4, & -0.9 < x \leq 0.17 \\
          0.45, & 0.17 < x \leq 0.24 \\
          0.5, & 0.24 < x \leq 0.34 \\
          0.55, & 0.34 < x \leq 0.38 \\
          0.6, & 0.38 < x \leq 0.52 \\
          0.7, & 0.52 < x \leq 0.55 \\
          0.75, & 0.55 < x \leq 0.62 \\
          0.8, & 0.62 < x \leq 0.73 \\
          0.85, & 0.73 < x \leq 0.8 \\
          0.9, & 0.8 < x \leq 0.9 \\
          0.95, & 0.9 < x \leq 1.31 \\
          1 & 1.31 < x
        \end{cases}
      \]
    
    \item Гистрограмма и полигон приведенных частот группированной выборки приведены на
    рисунках \ref{fig:polygon} и \ref{fig:hist} соответственно.
  \end{itemize}
  Листинг программы представлен в конце отчета.

  \begin{figure}[ht]
    \centering
    \includegraphics[width=0.95\textwidth]{emp_func.pdf}
    \caption{График эмпирической функции распределения}\label{fig:emp}
  \end{figure}

  \begin{figure}[ht]
    \centering
    \includegraphics[width=0.95\textwidth]{polygon.pdf}
    \caption{Полигон приведенных частот группированной выборки}\label{fig:polygon}
  \end{figure}

  \begin{figure}[ht]
    \centering
    \includegraphics[width=0.95\textwidth]{hist.pdf}
    \caption{Гистограмма приведенных частот группированной выборки}\label{fig:hist}
  \end{figure}
  
  % \begin{table}
  %   \centering
  %   \begin{tabular}{| c | c |}
  %       \hline
  %       $x$ & $F_{20}^*(x)$ \\
  %       \hline
  %       -1.55 & 0.0 \\
  %       \hline
  %       -1.45 & 0.05 \\
  %       \hline
  %       -1.38 & 0.15 \\
  %       \hline
  %       -1.31 & 0.2 \\
  %       \hline
  %       -1.14 & 0.25 \\
  %       \hline
  %       -1.0 & 0.3 \\
  %       \hline
  %       -0.9 & 0.35 \\
  %       \hline
  %       0.17 & 0.4 \\
  %       \hline
  %       0.24 & 0.45 \\
  %       \hline
  %       0.34 & 0.5 \\
  %       \hline
  %       0.38 & 0.55 \\
  %       \hline
  %       0.52 & 0.6 \\
  %       \hline
  %       0.55 & 0.7 \\
  %       \hline
  %       0.62 & 0.75 \\
  %       \hline
  %       0.73 & 0.8 \\
  %       \hline
  %       0.8 & 0.85 \\
  %       \hline
  %       0.9 & 0.9 \\
  %       \hline
  %       1.31 & 0.95 \\
  %       \hline
  %       1.32 & 1.0 \\
  %       \hline
  %   \end{tabular}
  %   \caption{Эмпирическая функция распределения}\label{table:emp}
  % \end{table}
  %

  \begin{table}
    \centering
    \begin{tabular}{| c | c | c |}
        \hline
        $x$ & $n_i$ & $\frac{n_i}{n}$ \\
        \hline
        -1.55 & 1 & 0.05 \\
        \hline
        -1.45 & 2 & 0.1 \\
        \hline
        -1.38 & 1 & 0.05 \\
        \hline
        -1.31 & 1 & 0.05 \\
        \hline
        -1.14 & 1 & 0.05 \\
        \hline
        -1.0 & 1 & 0.05 \\
        \hline
        -0.9 & 1 & 0.05 \\
        \hline
        0.17 & 1 & 0.05 \\
        \hline
        0.24 & 1 & 0.05 \\
        \hline
        0.34 & 1 & 0.05 \\
        \hline
        0.38 & 1 & 0.05 \\
        \hline
        0.52 & 2 & 0.1 \\
        \hline
        0.55 & 1 & 0.05 \\
        \hline
        0.62 & 1 & 0.05 \\
        \hline
        0.73 & 1 & 0.05 \\
        \hline
        0.8 & 1 & 0.05 \\
        \hline
        0.9 & 1 & 0.05 \\
        \hline
        1.31 & 1 & 0.05 \\
        \hline
    \end{tabular}
    \caption{Статистический ряд распределения}\label{table:var}
  \end{table}
  
  

  \clearpage
  \inputminted{python}{prog.py}

\end{document}
