\section{Текст задания}
\begin{enumerate}
    \item Обязательное задание. Написать программу на языке Python 3.x, которая бы осуществляла
    парсинг и конвертацию исходного файла в новый.
    \item Дополнительное задание \textnumero 1.
          \begin{enumerate}
              \item Найти готовые библиотеки, осуществляющие аналогичный парсинг и конвертацию файлов.
              \item Переписать исходный код, применив найденные библиотеки.
                    Регулярные выражения также нельзя использовать.
              \item Сравнить полученные результаты и объяснить их сходство/различие
          \end{enumerate}
    \item Дополнительное задание \textnumero 2.
    \begin{enumerate}
        \item Переписать исходный код, добавив в него использование регулярных выражений.
        \item Сравнить полученные результаты и объяснить их сходство/различие.
    \end{enumerate}
    \item Дополнительное задание \textnumero 3.
    \begin{enumerate}
        \item Используя свою исходную программу из обязательного задания, программу из дополнительного задания \textnumero 1 и
        программу из дополнительного задания \textnumero 2, сравнить стократное время выполнения парсинга + конвертации в цикле.
        \item Проанализировать полученные результаты и объяснить их сходство/различие.
    \end{enumerate}
    \item Дополнительное задание \textnumero 4.
    \begin{enumerate}
        \item Переписать исходную программу, чтобы она осуществляла
        парсинг и конвертацию исходного файла в любой другой
        формат (кроме JSON, YAML, XML, HTML): PROTOBUF,
        TSV, CSV, WML и т.п.
        \item Проанализировать полученные результаты, объяснить
        особенности использования формата
    \end{enumerate}
\end{enumerate}