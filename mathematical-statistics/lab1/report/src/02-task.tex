\section{Задание 1}
Выберите распределние, у которого существуют первые четыре момента,
и экспериментально убедитесь в асимптотической нормальности
выборочного среднего, выборочной дисперсии, выборочной квантили
порядка 0.5 для данного распределения.
Также экспериментально убедитесь в том, что 
\(n F(X_{(2)}) \to U_1 \sim \Gamma(2, 1) \) и
\( n(1 - F(X_{(n)})) \to U_2 \sim \Gamma(1, 1) = Exp(1) \).

\subsection{Алгоритм решения}
Для решения сгенерируем 1000 случайных выборок
размера 1000.
Каждый элемент выборки равномерно распределен от 0 до 1.
Для нахождения \(k\)-й порядоковой статистики отсортируем
массив и возьмем в нем \(k\)-й элемент.
Для построения графика нормального распределения \(\Gamma(2,1)\)
воспользуемся функцией \texttt{scipy.stats.gamma.pdf}.
Сравнение гистограммы распределения и графика нормального
распределения представлены на рисунке~\ref{fig:1_1}.

Аналогичным способом построим график для сравнения 
распределения с экспоненциальным.
Для построения графика экспоненциального распределения
воспользуемся функцией \texttt{scipy.stats.expon.pdf}.
Сравнение представлено на рисунке~\ref{fig:1_2}

Графики распределения среднего, дисперсии,
третьего и четверго момента и медианы представлены
на рисунках~\ref{fig:1_mean}, \ref{fig:1_variance}, \ref{fig:1_moments_3}, \ref{fig:1_moments_4}, \ref{fig:1_median} соответственно.

\subsection{Графики}

\begin{figure}[H]
  \centering
    \includegraphics[width=0.95\textwidth]{figures/1_1.pdf}
  \caption{Сравнение гистограммы распределения и графика нормального
распределения}\label{fig:1_1}
\end{figure}

\begin{figure}[H]
  \centering
    \includegraphics[width=0.95\textwidth]{figures/1_2.pdf}
  \caption{Сравнение гистограммы распределения и графика экспоненциального распределения}\label{fig:1_2}
\end{figure}

\begin{figure}[H]
  \centering
    \includegraphics[width=0.95\textwidth]{figures/1_mean.pdf}
  \caption{График распределения среднего}\label{fig:1_mean}
\end{figure}


\begin{figure}[H]
  \centering
    \includegraphics[width=0.95\textwidth]{figures/1_variance.pdf}
  \caption{График распределения дисперсии}\label{fig:1_variance}
\end{figure}


\begin{figure}[H]
  \centering
    \includegraphics[width=0.95\textwidth]{figures/1_moments_3.pdf}
  \caption{График распределения третьего момента}\label{fig:1_moments_3}
\end{figure}


\begin{figure}[H]
  \centering
    \includegraphics[width=0.95\textwidth]{figures/1_moments_4.pdf}
  \caption{График распределения четвертого момента}\label{fig:1_moments_4}
\end{figure}


\begin{figure}[H]
  \centering
    \includegraphics[width=0.95\textwidth]{figures/1_median.pdf}
  \caption{График распределения медианы}\label{fig:1_median}
\end{figure}

\subsection{Код программы}
\inputminted[breaklines,linenos]{Python}{py/task1.py}

\subsection{Описание результата}
Видно, что гистограмма \(n F(X_{(2)}))\) визуальна схожа
с графиком распределения \(n \Gamma(2,1)\).
Очевидно, что \(n \Gamma(2,1)\) имеет нормальное распределение,
а значит можно утверждать, что \(n F(X_{(2)}) \to n \Gamma(2,1) \sim \Gamma(2,1)\).

Аналогично можно утверждать, что \( n(1 - F(X_{(n)})) \to \Gamma(1,1) \).
Кроме того видно, что гистограммы распределения первых четырех моментов и
медианы выборок имеют нормальное распределение.

\section{Задание 2}
В файле \texttt{iris.csv} представлены данные о параметрах различных
экземплярах цветка ириса.
Какой вид в датасете представлен больше всего, какой -- меньше?
Рассчитайте выборочное среднее, выборочную дисперсию,
выборочную медиану и выборочную квантиль порядка 2/5 для
суммарной площади чашелистика и лепестка всей совокупности и
отдельно для каждого вида.
Построить график эмпирической функции распределения, гистограму
и box-plot суммарной площади чашелистика и лепестка для всей
совокупности и каждого вида.

Значения выборочного среднего, дисперсии, медиана и квантили порядка
2/5 для суммарной площади чашелистика и лепестка всей совокупности
и отдельно для каждого вида представлены в таблице~\ref{table:sepal_petal}.

График эмпирической функции распределения, гистограму
и box-plot суммарной площади чашелистика и лепестка для всей
совокупности и каждого вида представлены на рисунке~\ref{fig:sepal_petal}.

Для расчета значений и построения графиков были использованы
методы \texttt{numpy.mean}, \texttt{numpy.var}, \texttt{numpy.median}, \texttt{np.quantile} библиотеки
numpy и методы \texttt{seaborn.kdeplot}, \texttt{seaborn.histplot} и \texttt{seaborn.boxplot} библиотеки seaborn.

\begin{table}[H]
  \centering
    \begin{tabular}[c]{|l|r|r|r|r|r|}
      \hline
      Вид & Количество & Среднее & Дисперсия & Медиана & Квантиль 2/5 \\
      \hline
      All & 150 & 23.6169 & 47.5901 & 22.5000 & 20.3160 \\
      \hline
      setosa & 50 & 17.6234 & 8.7613 & 17.6600 & 16.7360 \\
      \hline
      versicolor & 50 & 22.2466 & 15.5174 & 22.2100 & 21.1420 \\
      \hline
      virginica & 50 & 30.9808 & 26.4648 & 31.4750 & 29.7160 \\
      \hline
    \end{tabular}
    \caption{Значения среднего, дисперсии, медиана, квантиль 2/5}\label{table:sepal_petal}
\end{table}


\begin{figure}[H]
  \centering
  \includegraphics[width=\textwidth]{figures/2_sepal_petal.pdf}
  \caption{График эмпирической функции распределения, гистограму
и box-plot суммарной площади чашелистика и лепестка для всей
  совокупности и каждого вида}\label{fig:sepal_petal}
\end{figure}

\subsection{Исходный код}
\inputminted[breaklines,linenos]{Python}{py/task2.py}

\subsection{Результаты}
График <<ящик с усами>> дает наиболее полное представление о датасете. Проанализируем его.
Видно, что по всей выборки суммарная площадь достаточно сильно распределена от 10 до 44.
При этом видно, что есть некоторые выбросы сверху.
Можно заметить, что в медианное значение площади сильно зависит от вида,
значения квантили 1/4 и 3/4 также сильно зависят от вида.
Выбросы в общей картине наблюдаются исключительно из-за третьего вида <<virginica>>.


\section{Вывод}
В ходе выполнения данной лабораторной работы
мною были изучены нормальное и экспоненциальное распределение,
я на практике изучил основы работы с статистическими данными
в языке Python с помощью библиотек pandas, numpy, scipy, matplotlib и seaborn.
