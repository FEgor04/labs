\documentclass{article}
\usepackage[T2A]{fontenc}
\usepackage[russian]{babel}
\usepackage{float}
\usepackage{mathtools}
\usepackage{gensymb}
\usepackage{textcomp}
\usepackage{tabularray}
\usepackage{csquotes}
\usepackage{fixmath}
\usepackage{amsmath}
\usepackage{amssymb}
\usepackage{arcs}
\usepackage{wasysym}
\usepackage{hyperref}
\usepackage[russian]{cleveref}

\everymath{\displaystyle}
\let\Im\relax
\let\Re\relax
\DeclareMathOperator{\Re}{Re}
\DeclareMathOperator{\Im}{Im}

\DeclareMathOperator{\grad}{grad}
\DeclareMathOperator{\rot}{rot}
\DeclareMathOperator{\Div}{div}

\renewcommand{\vec}[1]{\mathbold{#1}}

\newcommand{\Real}{\mathbb{R}}
\newcommand{\Complex}{\mathbb{C}}

\usepackage{tikz}

\usepackage[left=2cm,right=2cm,top=2cm,bottom=2cm]{geometry}


\title{ДЗ по матстату}
\author{Егор Федоров, P3215}
\date{07.03.2024}

\begin{document}
\maketitle

\section{Задача 1.8}
% Упростить выражение \(A = (B+C)(B + \bar C)(\bar B + C)\).
\[
	A =
	(B + C) (B + \bar C) (\bar B + C) =
	B (\bar B + C) =
	B \bar B + B C =
	\varnothing + B C =
	B C
\]
\section{Задача 2.7}
% \textit{
%   Из партии деталей, среди которых \(n\) доброкачественных
%   и \(m\) бракованных, для контроля наудачу взято \(s\) штук.
%   При контроле оказалось, что первые \(k\) из \(s\) деталей 
%   доброкачественные.
%   Определить вероятность того, что следующая деталь будет доброкачественной.
% }

После того как взяли \(s\) деталей, из которых \(k\) -- доброкачественные,
осталось \(n - k\) доброкачественных и \(m - (s - k)\) бракованных.
Тогда всего осталось \(m + n - s\) деталей, из которых
доброкачественных --- \( n-k \).
Тогда итоговая вероятность: \(P = \frac{n-k}{m+n-s}\).

\textbf{Ответ:} \(P = \frac{n-k}{m+n-s}\).

\section{Задача 3.11}
Зададим двумерное пространство событий.
Пусть \(a, b\) -- координаты получившихся точек, \(a \leq b\).
Тога \(a, b \leq l\), при этом \(a \leq 1/2 \cdot l \) и
\(b \geq 1/2 \cdot l\).
Тогда можно добавить ограничение на третью сторону, лежащуюу
между точками \(a\) и \(b\): \( b - a < 1/2 \cdot l \).

Получаем множество возможных исходов:
\[
	0 \leq a \leq l \qquad 0 \leq b \leq l \qquad
	a \leq b
\]
Начертим на графике и получим прямоугольный треугольник
с катетами \(l\).
Его площадь -- \( 1/2 \cdot l^2 \).

Множество благоприятных исходов:
\[
	a \leq 1/2 \cdot l
	\qquad
	b \geq 1/2 \cdot l
	\qquad
	b \leq 1/2 \cdot l + a
\]
Начертим на графике и получим прямоугольный треугольник
с катетами \(l/2\). Его площадь -- \(1/8 l^2\).

Тогда искомая вероятность:
\[
	P = \frac{1/2 \cdot l^2}{1/8 \cdot l^2} = 1/4
\]

\section{Задача 4.5}
Сначала найдем вероятность того, что цепь будет работать.
Для этого необходимо, чтобы работал первый элемент и работали второй или третий.

Вероятность того, что будет работать первый:
\[ P_1 = 1 - 0.3 = 0.7 \]
Вероятность того, что будет работать второй или третий:
\[ P_2 = 1 - 0.2 \cdot 0.2 = 0.96 \]

Тогда вероятность того, что цепь будет работать
\[ P' = P_1 \cdot P_2 = 0.7 \cdot 0.96 = 0.672  \]

Значит вероятность того, что цепь не будет работать:
\[ P = 1 - P' = 0.328 \]

\textbf{Ответ:} \( P = 0.328 \)

\section{Задача 5.4}
Пусть события \(A\) и \(B\) --- точка упадет на первую или вторую монету.

Тогда \(P(A + B) = P(A) + P(B) - P(AB)\).
Так как по условию монеты не перекрываются, то
\(P(AB) = 0\).
Тогда \(P(A+B) = P(A) + P(B)\).
А так как радиусы монет одинаковы, то
\(P(A+B) = 2 P(A)\).

Благоприятные исходы -- точка упадет на монету.
Площадь монеты равна \(\pi r^2\).

Возможные исходы -- точка упадет в круг.
Площадь круга равна \( \pi R^2 \).

Тогда искомая вероятность:
\[ P(A + B) = 2 P (A) = 2 \frac{\pi r^2}{\pi R^2} = 2 \left(\frac{r}{R}\right)^2 \]

\textbf{Ответ:} \(P = 2 \left( \frac{r}{R} \right)^2 \).

\section{Задача 6.3}
Если мы достали белый шар, то мы могли
либо достать белый шар из первой урны,
либо достать белый шар из второй урны.

Вероятность первого события:
\[ P_1 = \frac{1}{2} \cdot \frac{m_1}{m_1 + n_1} \]
Вероятность второго события:
\[ P_2 = \frac{1}{2} \cdot \frac{m_2}{m_2 + n_2} \]

Тогда итоговая вероятность
\[ P = P_1 + P_2 = \frac{1}{2} \left( \frac{m_1}{m_1 + n_1} + \frac{m_2}{m_2 + n_2} \right) \]

\textbf{Ответ:}
\( P = \frac{1}{2} \left( \frac{m_1}{m_1 + n_1} + \frac{m_2}{m_2 + n_2} \right) \)

\section{Задача 7.2}
Пусть гипотеза \(H_1\) --- <<шар извлечен из урны, принадлежащей первой группе>>,
\(H_2 = \bar H_1 \) --- <<шар извлечен из урны, принадлежащей второй группе>>.
Событие \(A\) --- <<шар -- белый>>.
Тогда:
\[ P(H_1) = \frac{k_1}{k_1 + k_2} \]
\[ P(A \mid H_1) = \frac{m_1}{m_1 + n_1} \]
\[ P(A) = P(H_1) \cdot P(A \mid H_1) + P(H_2) \cdot P(A \mid H_2) \]

По формуле Байеса:
\begin{align*}
	P(H \mid A) & =
	\frac{P(A \mid H_1) \cdot P(H_1)}{P(A)} =
	\frac{P(A \mid H_1) \cdot P(H_1)}{P(H_1) \cdot P(A \mid H_1) + P(H_2) \cdot P(A \mid H_2)} =                      \\
	            & =
	1 - \frac{P(H_2) \cdot P(A \mid H_2)}{P(H_1) \cdot P(A \mid H_1)} =
	1 - \frac{ \frac{k_2}{k_1+k_2} \cdot \frac{m_2}{m_2 + n_2} }{\frac{k_1}{k_1 + k_2} \cdot \frac{m_1}{m_1 + n_1}} = \\
	            & =
	1 - \frac{k_2 m_2}{(k_1 + k_2)(m_2 + n_2)} \cdot \frac{(k_1 + k_2) (m_1 + n_1)}{k_1 m_1} =                        \\
	            & =
	1 - \frac{k_2 m_2 (m_1 + n_1)}{k_1 m_1 (m_2 + n_2)}
\end{align*}

\textbf{Ответ:} \(P = 1 - \frac{k_2 m_2 (m_1 + n_1)}{k_1 m_1 (m_2 + n_2)} \)

\section{Задача 8.1}
\begin{enumerate}
	\item не содержит цифры пять:
	      \[ \left(\frac{9}{10}\right)^4 = 0.6561 \]
	\item не содержит двух и более пятерок (т.е. содержит 0 или 1 пятерку):
	      \[
		      0.9^4 + C_4^1 \cdot 0.1 \cdot 0.9^3 = 0.9477
	      \]
	\item не содержит ровно двух пятерок:
	      \[
		      1 - \left(C_4^2 \cdot 0.1^2 \cdot 0.9^2 \right) =
		      1 - 6 \cdot 0.1^2 \cdot 0.9^2 =
		      0.9514
	      \]
\end{enumerate}

\end{document}
