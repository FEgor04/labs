\section{Текст задания}
Для выполнения лабораторной работы №1 необходимо:

\begin{enumerate}
    \item На основе предложенной предметной области (текста) составить ее описание. Из полученного описания выделить сущности, их атрибуты и связи.
    \item Составить инфологическую модель.
    \item Составить даталогическую модель. При описании типов данных для атрибутов должны использоваться типы из СУБД PostgreSQL.
    \item Реализовать даталогическую модель в PostgreSQL. При описании и реализации даталогической модели должны учитываться ограничения целостности, которые характерны для полученной предметной области.
    \item Заполнить созданные таблицы тестовыми данными.
\end{enumerate}

\textbf{Описание предметной области, по которой должна быть построена доменная модель:}
\begin{quote}
    Хэммонд брал его с собой на все эти переговоры по сбору денег в свой Фонд.
    Как правило, Дженнаро вносил в комнату клетку, покрытую небольшим одеялом,
    как заварочный чайник с чехлом, и Хэммонд произносил свою обычную речь о
    перспективах развития так называемой "биологической потребительской продукции".
    Затем в самый патетический момент речи он срывал одеяло с клетки,
    и слон представал перед глазами собравшихся. И тут Хэммонд просил денег.
\end{quote}